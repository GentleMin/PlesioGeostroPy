\section{Nondimensionalizations, conversions, background field scaling}

In this section I discuss three dimensionless forms of the governing equations, and their conversions.
In the basic MHD in rotating frame setting, the governing equations are the Navier Stokes equation (with Coriolis force and Lorentz force) and the magnetic induction equation,
\begin{equation}
\begin{aligned}
    & \frac{\partial \mathbf{u}}{\partial t} + \mathbf{u}\cdot \nabla \mathbf{u} + 2\Omega \hat{\mathbf{z}}\times \mathbf{u} = -\nabla \frac{P}{\rho} + \frac{1}{\rho \mu_0}(\nabla\times\mathbf{B})\times \mathbf{B} + \nu \nabla^2 \mathbf{u} \\ 
    & \frac{\partial \mathbf{B}}{\partial t} = \nabla\times (\mathbf{u}\times \mathbf{B}) + \eta \nabla^2 \mathbf{B}
\end{aligned}
\end{equation}
These equations only couple the kinematic quantities (i.e. velocity) and magnetic quantities (i.e. magnetic field). Therefore, four scales are needed to nondimensionalize the equations. These quantities are two scales associated with the dynamic variables, the velocity scale $U$ and the magnetic field scale $\mathscr{B}$, and two scales associated with the spatial/evolution variables, the spatial scale $L$ and the time scale $\tau$. Among these, only three are independent, as the velocity scale is usually linked to the time scale and the spatial scale via $U = L/\tau$. The three independent scales are therefore the length scale $L$, the time scale $\tau$, and the magnetic field scale $\mathscr{B}$. As we are considering a system where the inertial effect dominates, the pressure scale is chosen to be $\Pi = \rho U^2$ (as Jerome would put it, pressure is the slave of velocity). Rewriting the equations in dimensionless quantities using the aforementioned scales, we have
\[
\begin{aligned}
    & \frac{\partial \mathbf{u}}{\partial t} + \mathbf{u}\cdot \nabla \mathbf{u} + \nabla p + 2\Omega \tau \hat{\mathbf{z}}\times \mathbf{u} = \frac{\mathscr{B}^2 \tau^2}{\rho \mu_0 L^2}(\nabla\times\mathbf{B})\times \mathbf{B} + \frac{\nu \tau}{L^2} \nabla^2 \mathbf{u} \\ 
    & \frac{\partial \mathbf{B}}{\partial t} = \nabla\times (\mathbf{u}\times \mathbf{B}) + \frac{\eta\tau}{L^2} \nabla^2 \mathbf{B}
\end{aligned}
\]

A common choice of the length scale is given by the radius of the sphere, i.e. $L = R$. The difference lies in the choice of time scale. In a diffusive MHD system in a rotating frame as such, there are four time scales at play here. First, due to the rotation, one can define the rotation time scale:
\[
    \tau_\Omega = \Omega^{-1}.
\]
This is the time scale that characterizes the period of rotation (apart from a $2\pi$ factor). Next, due to the presence of magnetic field, an Alfvén time scale can be defined from the scale of the magnetic field
\[
    \tau_A = \frac{\sqrt{\rho \mu_0} L}{\mathscr{B}}.
\]
This is defined by $\tau_A = L/V_A$, where $V_A = \mathscr{B}/\sqrt{\rho \mu_0}$ is the Alfvén wave velocity associated with a magnetic field with magnitude $\mathscr{B}$.
Thirdly, due to the finite conductivity, there is a characteristic magnetic diffusion time scale
\[
    \tau_\eta = \frac{L^2}{\eta}.
\]
And finally due to viscosity, there is a characteristic viscous diffusion time scale
\[
    \tau_\nu = \frac{L^2}{\nu}.
\]
In the regime of Earth's core, these time scales are separated from one another by at least three orders of magnitude, if not even more. The relative magnitudes of these time scales are 
\begin{equation}
    \tau_\Omega \left(\sim 10^4 \mathrm{s}\right) \ll \tau_A \left(\sim 10^1 \mathrm{yr}\right) \ll \tau_\eta \left(\sim 10^5 \mathrm{yr}\right) \ll \tau_\nu \left(\sim 10^{10} \mathrm{yr}\right).
\end{equation}
The actual core flow (as e.g. estimated from surface flow inversion) induces a circulation time scale that is somewhere in between the Alfvén time scale and the magnetic diffusion time scale, closer to the former. It is approx. one order of magnitude larger than the Alfvén time scale, or the core flow velocity is one order of magnitude slower than the Alfvén velocity. That being said, in a turbulent system as the one in Earth's core, the observed flow velocity is likely dependent on the time scale where such observation takes place.

Using these time scales, the nondimensional equations can be rewritten as
\begin{equation}
    \begin{aligned}
        & \frac{\partial \mathbf{u}}{\partial t} + \mathbf{u}\cdot \nabla \mathbf{u} + \nabla p + 2\frac{\tau}{\tau_\Omega} \hat{\mathbf{z}}\times \mathbf{u} = \frac{\tau^2}{\tau_A^2}(\nabla\times\mathbf{B})\times \mathbf{B} + \frac{\tau}{\tau_\nu} \nabla^2 \mathbf{u} \\ 
        & \frac{\partial \mathbf{B}}{\partial t} = \nabla\times (\mathbf{u}\times \mathbf{B}) + \frac{\tau}{\tau_\eta} \nabla^2 \mathbf{B}
    \end{aligned}
\end{equation}
It now boils down to what to use for $\tau$, which is the key to different nondimensionalizations. Since the velocity scale is probably closest to Alfvén wave speed, the most common strategy seems to use $\tau = \tau_A$ as the time scale (see e.g. \cite{canet_hydromagnetic_2014}; \cite{holdenried-chernoff_long_2021}; \cite{luo_waves_2022}). The Lorentz force term will then have unit factor. The full equations take the form
\begin{equation}
    \begin{aligned}
        & \frac{\partial \mathbf{u}}{\partial t} + \mathbf{u}\cdot \nabla \mathbf{u} + \nabla p + \frac{2}{\mathrm{Le}} \hat{\mathbf{z}}\times \mathbf{u} = (\nabla\times\mathbf{B})\times \mathbf{B} + \frac{\mathrm{Pm}}{\mathrm{Lu}} \nabla^2 \mathbf{u} \\ 
        & \frac{\partial \mathbf{B}}{\partial t} = \nabla\times (\mathbf{u}\times \mathbf{B}) + \frac{1}{\mathrm{Lu}} \nabla^2 \mathbf{B}
    \end{aligned}
\end{equation}
where the prefactors of the Coriolis term, viscous diffusion term and magnetic diffusion term are described by the Lehnert number ($\mathrm{Le}$), the magnetic Prandtl number ($\mathrm{Pm}$), and the Lundquist number ($\mathrm{Lu}$), which are defined as
\begin{equation}
\begin{aligned}
    \mathrm{Le} &= \frac{\tau_\Omega}{\tau_A} = \frac{V_A}{V_\Omega} = \frac{\mathscr{B}}{\sqrt{\rho \mu_0}\Omega L}, \\ 
    \mathrm{Pm} &= \frac{\tau_\eta}{\tau_\nu} = \frac{\nu}{\eta}, \\
    \mathrm{Lu} &= \frac{\tau_\eta}{\tau_A} = \frac{\mathscr{B} L}{\sqrt{\rho \mu_0}\eta}.
\end{aligned}
\end{equation}

The second nondimensionalization uses the rotation time as the time scale, i.e. $\tau = \tau_\Omega$ \parencite{jackson_plesio-geostrophy_2020}. This is particularly useful in abscence of magnetic field, when a purely hydromagnetic system is considered. The equations then read
\begin{equation}
    \begin{aligned}
        & \frac{\partial \mathbf{u}}{\partial t} + \mathbf{u}\cdot \nabla \mathbf{u} + \nabla p + 2\hat{\mathbf{z}}\times \mathbf{u} = \mathrm{Le}^2 (\nabla\times\mathbf{B})\times \mathbf{B} + E \nabla^2 \mathbf{u} \\ 
        & \frac{\partial \mathbf{B}}{\partial t} = \nabla\times (\mathbf{u}\times \mathbf{B}) + E_\eta \nabla^2 \mathbf{B}
    \end{aligned}
\end{equation}
where the prefactors of the Lorentz force term, viscous diffusion term and magnetic diffusion term are described by the Lehnert number ($\mathrm{Le}$), the Ekman number ($E$), and the magnetic Ekman number ($E_\eta$). The latter two are defined as
\begin{equation}
\begin{aligned}
    E &= \frac{\tau_\Omega}{\tau_\nu} = \frac{\nu}{\Omega L^2} = \mathrm{Pm} E_\eta, \\
    E_\eta &= \frac{\tau_\Omega}{\tau_\eta} = \frac{\eta}{\Omega L^2}.
\end{aligned}
\end{equation}

The third nondimensionalization uses the magnetic diffusion time as the time scale, i.e. $\tau = \tau_\eta$ \parencite{luo_waves2_2022}. This might be more useful when describing dynamo actions that sustain the field over long time scales. The equations read
\[
    \begin{aligned}
        & \frac{\partial \mathbf{u}}{\partial t} + \mathbf{u}\cdot \nabla \mathbf{u} + \nabla p + \frac{2}{E_\eta} \hat{\mathbf{z}}\times \mathbf{u} = \frac{\Lambda}{E_\eta} (\nabla\times\mathbf{B})\times \mathbf{B} + \mathrm{Pm} \nabla^2 \mathbf{u} \\ 
        & \frac{\partial \mathbf{B}}{\partial t} = \nabla\times (\mathbf{u}\times \mathbf{B}) + \nabla^2 \mathbf{B}
    \end{aligned}
\]
or alternatively 
\begin{equation}
    \begin{aligned}
        & E_\eta \left(\frac{\partial \mathbf{u}}{\partial t} + \mathbf{u}\cdot \nabla \mathbf{u} + \nabla p\right) + 2 \hat{\mathbf{z}}\times \mathbf{u} = \Lambda (\nabla\times\mathbf{B})\times \mathbf{B} + E \nabla^2 \mathbf{u} \\ 
        & \frac{\partial \mathbf{B}}{\partial t} = \nabla\times (\mathbf{u}\times \mathbf{B}) + \nabla^2 \mathbf{B}
    \end{aligned}
\end{equation}
where the prefactors of the inertial term, Lorentz force term and the viscous diffusion term are described by the magnetic Ekman number ($E_\eta$), the Elsasser number $\Lambda$, and the Ekman number ($E$), respectively. The Elsasser number is defined as
\begin{equation}
\Lambda = \frac{\tau_\Omega \tau_\eta}{\tau_A^2} = \frac{\mathscr{B}^2}{\rho \mu_0 \eta \Omega}
\end{equation}
It is somehow confusing to interpret Elsasser number as ratio between different time scales. The more straightforward explanation is that it is merely a factor that measures the ratio of Lorentz to Coriolis forces, arising when time is measured in magnetic diffusion time scale.

\subsection{Conversion of dimensionless numbers}

\begin{table}[htbp]
\centering
\begin{tabular}[c]{p{2.5cm}|p{2.5cm}|p{2.5cm}|p{2.5cm}|p{2.5cm}}
    \toprule
    Dimless Params & Rotation tscale, $\mathrm{Le}$, $E_\eta$, $E$ & Alfvén tscale, $\mathrm{Le}$, $\mathrm{Lu}$, $\mathrm{Pm}$ & Diffusion tscale, $\Lambda$, $E_\eta$, $E$ & Diffusion div2, $\Lambda'$, $E_\eta'$, $E'$ \\
    \hline
    Rotation tscale, $\mathrm{Le}$, $E_\eta$, $E$ & / 
    & $\begin{aligned}
        \mathrm{Le} &= \mathrm{Le} \\ 
        E_\eta &= \mathrm{Le} / \mathrm{Lu} \\ 
        E &= \mathrm{Pm} \mathrm{Le} / \mathrm{Lu} \\
        t :&= t / \mathrm{Le} 
    \end{aligned}$
    & $\begin{aligned}
        \mathrm{Le} &= \sqrt{E_\eta \Lambda} \\ 
        E_\eta &= E_\eta \\ 
        E &= E \\
        t :&= t / E_\eta
    \end{aligned}$
    & $\begin{aligned}
        \mathrm{Le} &= 2\sqrt{E_\eta' \Lambda'} \\ 
        E_\eta &= 2E_\eta' \\ 
        E &= 2E' \\
        t :&= t / 2E_\eta'
    \end{aligned}$ \\
    \hline
    Alfvén tscale, $\mathrm{Le}$, $\mathrm{Lu}$, $\mathrm{Pm}$ 
    & $\begin{aligned}
        \mathrm{Le} &= \mathrm{Le} \\ 
        \mathrm{Lu} &= \mathrm{Le} / E_\eta \\ 
        \mathrm{Pm} &= E/E_\eta \\ 
        t :&= \mathrm{Le} \cdot t
    \end{aligned}$ & /
    & $\begin{aligned}
        \mathrm{Le} &= \sqrt{E_\eta \Lambda} \\ 
        \mathrm{Lu} &= \sqrt{\Lambda / E_\eta} \\ 
        \mathrm{Pm} &= E/E_\eta \\
        t :&= \sqrt{\Lambda / E_\eta} \, t
    \end{aligned}$ 
    & $\begin{aligned}
        \mathrm{Le} &= 2\sqrt{E_\eta' \Lambda'} \\ 
        \mathrm{Lu} &= \sqrt{\Lambda' / E_\eta'} \\ 
        \mathrm{Pm} &= E'/E_\eta' \\
        t :&= \sqrt{\Lambda' / E_\eta'} \, t
    \end{aligned}$ \\
    \hline
    Diffusion tscale, $\Lambda$, $E_\eta$, $E$ 
    & $\begin{aligned}
        \Lambda &= E_\eta \mathrm{Le}^2 \\ 
        E_\eta &= E_\eta \\ 
        E &= E \\
        t :&= E_\eta \, t
    \end{aligned}$ 
    & $\begin{aligned}
        \Lambda &= \mathrm{Le} \mathrm{Lu} \\ 
        E_\eta &= \mathrm{Le} / \mathrm{Lu} \\ 
        E &= \mathrm{Pm} \mathrm{Le} / \mathrm{Lu} \\
        t :&= t/\mathrm{Lu}
    \end{aligned}$ & / 
    & $\begin{aligned}
        \Lambda &= 2\Lambda' \\ 
        E_\eta &= 2E_\eta' \\
        E &= 2E' \\
        t :&= t
    \end{aligned}$ \\
    \hline
    Diffusion div2, $\Lambda'$, $E_\eta'$, $E'$ 
    & $\begin{aligned}
        \Lambda' &= E_\eta \mathrm{Le}^2/2 \\ 
        E_\eta' &= E_\eta/2 \\ 
        E' &= E/2 \\
        t :&= E_\eta \, t
    \end{aligned}$ 
    & $\begin{aligned}
        \Lambda' &= \mathrm{Le} \mathrm{Lu}/2 \\ 
        E_\eta' &= \mathrm{Le} / 2\mathrm{Lu} \\ 
        E' &= \mathrm{Pm} \mathrm{Le} / 2\mathrm{Lu} \\
        t :&= t/\mathrm{Lu}
    \end{aligned}$ 
    & $\begin{aligned}
        \Lambda' &= \Lambda/2 \\ 
        E_\eta' &= E_\eta/2 \\
        E' &= E/2 \\
        t :&= t
    \end{aligned}$ & / \\
    \bottomrule
\end{tabular}
\end{table}


\subsection{Operators form, linearizations}

Let us consider the aforementioned system, comprising of the momentum equation with Coriolis force, Lorentz force and viscous force, and the magnetic induction equation with magnetic diffusion. In the operator form, the system reads
\begin{equation}
    \frac{\partial}{\partial t} \begin{pmatrix} \alpha_u \mathcal{M}_u & \\ & \alpha_b \mathcal{M}_b \end{pmatrix}
    \begin{pmatrix} \hat{u} \\ \hat{B} \end{pmatrix} = 
    \begin{pmatrix}
        \alpha_C \mathcal{K}_C \hat{u} + \alpha_L \mathcal{F}_{L}(\hat{B}) + \alpha_\nu \mathcal{K}_\nu \hat{u} \\ 
        \alpha_I \mathcal{F}_{I}(\hat{u}, \hat{B}) + \alpha_\eta \mathcal{K}_\eta \hat{B}
    \end{pmatrix}
\end{equation}
where $\alpha$ with different subscripts give the possible dimensionless numbers in front of the respective terms. Among these terms, two concern in general nonlinear operators; these are the Lorentz force term $\mathcal{F}_L(\hat{B})$, and the induction term $\mathcal{F}_I(\hat{u}, \hat{B})$. However, the concerned operators are still homogeneous. For instance, the Lorentz operator is quadratic in the magnetic quantity:
\[
    \mathcal{F}_L(c\hat{B}) = c^2 \mathcal{F}_L(B)
\]
and the induction operator is bilinear in either the velocity or the magnetic quantity
\[
    \mathcal{F}_I(c\hat{u}, \hat{B}) = c \mathcal{F}_I(\hat{u}, \hat{B}) = \mathcal{F}_I(\hat{u}, c\hat{B}).
\]
In the eigenvalue problem, these operators are further linearised. The linearisation yields the linear operators under background field $\hat{B}_0$, $\hat{u}_0$,
\[
    \mathcal{K}_L(\hat{B}_0) = \mathsf{D}_b \mathcal{F}_L|_{\hat{B} = \hat{B}_0},\quad 
    \mathcal{K}_{I,u}(\hat{B}_0) = \mathsf{D}_u \mathcal{F}_I|_{\hat{B} = \hat{B}_0},\quad 
    \mathcal{K}_{I,b}(\hat{u}_0) = \mathsf{D}_b \mathcal{F}_I|_{\hat{u} = \hat{u}_0}
\]
which are further themselves linear in the background field. In case of zero background field, these operators also vanish.
They then make up the linearised system
\begin{equation}
    \frac{\partial}{\partial t} \begin{pmatrix} \alpha_u \mathcal{M}_u & \\ & \alpha_b \mathcal{M}_b \end{pmatrix}
    \begin{pmatrix} \hat{u} \\ \hat{b} \end{pmatrix} = 
    \begin{pmatrix}
        \alpha_C \mathcal{K}_C + \alpha_\nu \mathcal{K}_\nu & \alpha_L \mathcal{K}_{L}(\hat{B}_0) \\ 
        \alpha_I \mathcal{K}_{I,u}(\hat{B}_0) & \alpha_I \mathcal{K}_{I,b}(\hat{u}_0) + \alpha_\eta \mathcal{K}_\eta
    \end{pmatrix}
    \begin{pmatrix} \hat{u} \\ \hat{b} \end{pmatrix}.
\end{equation}

\medskip
Let us take a very brief detour and discuss a more general question:
how to establish the equivalence of the system with different nondimensionalisations in the operator form? Surely they are equivalent physically, and the only difference in their mathematical forms is the different scalings of time, length, forcing, etc.
As an example, let's consider two linear dynamical systems in the forms of 
\[\begin{aligned}
    \frac{\partial}{\partial t}\mathcal{M} \hat{x} = \mathcal{K}_1 \hat{x},\qquad 
    \frac{\partial}{\partial t}\mathcal{M} \hat{x} = \mathcal{K}_2 \hat{x}
\end{aligned}\]
where $\hat{x} \in \mathcal{V}^n$ is the vector of dynamical variables, and $\mathcal{M}$ and $\mathcal{K}$ are linear maps $\mathcal{V}^n \mapsto \mathcal{V}^n$. In addition, we require that the mass operator is diagonal. 
The two dynamical systems can be transformed into one another with a simple change of scaling \textit{iff} $\exists$ a invertible diagonal transform $\mathbf{D} \in \mathbb{R}^n$ such that the operators $\mathcal{K}_1$ and $\mathcal{K}_2$ are similar to one another under this transform, up to a scalar factor. In other words,
\[
    \exists \mathbf{d} \in \mathbb{R}^n (d_i\neq 0), \mathbf{D} = \mathrm{diag}(\mathbf{d}),\quad \exists \tau_0 \in \mathbb{R}, \quad 
    s.t.\quad \mathcal{K}_2 = \tau_0 \mathbf{D}^{-1} \mathcal{K}_1 \mathbf{D}.
\]
The $\tau_0$ factor will give the ratio between two time scales in the two systems.
Indeed, using this transform, the second system can be rewritten as
\[
    \frac{\partial}{\partial t} \mathcal{M} \hat{x} = \tau_0 \mathbf{D}^{-1} \mathcal{K}_1 \mathbf{D} \hat{\mathbf{x}} \quad \Longleftrightarrow \quad 
    \frac{\partial}{\partial (\tau_0 t)} \mathcal{M} \left(\mathbf{D} \hat{x}\right) = \mathcal{K}_1 \left(\mathbf{D} \hat{\mathbf{x}}\right).
\]
which is equivalent to the system one with rescaled dynamical variables $\hat{\mathbf{x}}' = \mathbf{D} \hat{\mathbf{x}}$ and rescaled time $t' = \tau_0 t$.
This provides another, perhaps a more general way to derive the change of variables and rescaling of time from one dimensionless form to the other.
Using the Alfvén time as the time scale, the linearized system takes the form
\begin{equation}
    \frac{\partial}{\partial t} \begin{pmatrix} \mathcal{M}_u & \\ & \mathcal{M}_b \end{pmatrix}
    \begin{pmatrix} \hat{u} \\ \hat{b} \end{pmatrix} = 
    \begin{pmatrix}
        \mathrm{Le}^{-1} \mathcal{K}_C + \mathrm{Pm} \mathrm{Lu}^{-1} \mathcal{K}_\nu & \mathcal{K}_{L}(\hat{B}_0) \\ 
        \mathcal{K}_{I,u}(\hat{B}_0) & \mathcal{K}_{I,b}(\hat{u}_0) + \mathrm{Lu}^{-1} \mathcal{K}_\eta
    \end{pmatrix}
    \begin{pmatrix} \hat{u} \\ \hat{b} \end{pmatrix}
\end{equation}
Using the rotation time as the time scale, the linearized system takes the form
\begin{equation}
    \frac{\partial}{\partial t} \begin{pmatrix} \mathcal{M}_u & \\ & \mathcal{M}_b \end{pmatrix}
    \begin{pmatrix} \hat{u} \\ \hat{b} \end{pmatrix} = 
    \begin{pmatrix}
        \mathcal{K}_C + E \mathcal{K}_\nu & \mathrm{Le}^2 \mathcal{K}_{L}(\hat{B}_0) \\ 
        \mathcal{K}_{I,u}(\hat{B}_0) & \mathcal{K}_{I,b}(\hat{u}_0) + E_\eta \mathcal{K}_\eta
    \end{pmatrix}
    \begin{pmatrix} \hat{u} \\ \hat{b} \end{pmatrix}
\end{equation}
Using the magnetic diffusion time as the time scale, the linearized system takes the form
\begin{equation}
    \frac{\partial}{\partial t} \begin{pmatrix} \mathcal{M}_u & \\ & \mathcal{M}_b \end{pmatrix}
    \begin{pmatrix} \hat{u} \\ \hat{b} \end{pmatrix} = 
    \begin{pmatrix}
        E_\eta^{-1} \mathcal{K}_C + \mathrm{Pm} \mathcal{K}_\nu & E_\eta^{-1} \Lambda \mathcal{K}_{L}(\hat{B}_0) \\ 
        \mathcal{K}_{I,u}(\hat{B}_0) & \mathcal{K}_{I,b}(\hat{u}_0) + \mathcal{K}_\eta
    \end{pmatrix}
    \begin{pmatrix} \hat{u} \\ \hat{b} \end{pmatrix}
\end{equation}


\subsection{Rescaling the background field}

\textcite{holdenried-chernoff_long_2021} compares the PG results with the 3-D results, calculated using the code from \textcite{luo_waves2_2022}. It is however very peculiar that the background fields used in their codes are not entirely consistent. While in the 3-D magneto-Coriolis eigenvalue problem code \parencite{luo_waves2_2022}, the background fields (for T1 and S1 fields) take a prefactor to approximately satisfy some normalization conditions, this is not the case for the PG code.
The question then is how the two results are comparable to each other. In other words, what would introducing a prefactor in the background magnetic field cause?

We focus on the dimensionless form with parameters $\Lambda$, $E_\eta$ and $E$. Consider a background field $\hat{B}_0' = \alpha \hat{B}_0$. Using the fact that the linearised operators are themselves linear functional of the background field, $\mathcal{K}_L(\alpha \hat{B}_0) = \alpha \mathcal{K}_L(\hat{B}_0)$ and $\mathcal{K}_{I,u}(\alpha \hat{B}_0) = \alpha \mathcal{K}_{I,u}(\hat{B}_0)$.
The system is
\[
    \frac{\partial}{\partial t} \begin{pmatrix} \mathcal{M}_u & \\ & \mathcal{M}_b \end{pmatrix}
    \begin{pmatrix} \hat{u} \\ \hat{b} \end{pmatrix} = 
    \begin{pmatrix}
        E_\eta^{-1} \mathcal{K}_C + \mathrm{Pm} \mathcal{K}_\nu & \alpha E_\eta^{-1} \Lambda \mathcal{K}_{L}(\hat{B}_0) \\ 
        \alpha \mathcal{K}_{I,u}(\hat{B}_0) & \mathcal{K}_{I,b}(\hat{u}_0) + \mathcal{K}_\eta
    \end{pmatrix}
    \begin{pmatrix} \hat{u} \\ \hat{b} \end{pmatrix}
\]
Introducing the corresponding rescaled perturbational magnetic field $\hat{b} = \alpha \hat{b}'$, we have 
\[\begin{aligned}
    \frac{\partial}{\partial t} \begin{pmatrix} \mathcal{M}_u & \\ & \alpha \mathcal{M}_b \end{pmatrix}
    \begin{pmatrix} \hat{u} \\ \hat{b}' \end{pmatrix} &= 
    \begin{pmatrix}
        E_\eta^{-1} \mathcal{K}_C + \mathrm{Pm} \mathcal{K}_\nu & \alpha^2 E_\eta^{-1} \Lambda \mathcal{K}_{L}(\hat{B}_0) \\ 
        \alpha \mathcal{K}_{I,u}(\hat{B}_0) & \alpha\left(\mathcal{K}_{I,b}(\hat{u}_0) + \mathcal{K}_\eta\right)
    \end{pmatrix}
    \begin{pmatrix} \hat{u} \\ \hat{b}' \end{pmatrix} \\ 
    \frac{\partial}{\partial t} \begin{pmatrix} \mathcal{M}_u & \\ & \mathcal{M}_b \end{pmatrix}
    \begin{pmatrix} \hat{u} \\ \hat{b}' \end{pmatrix} &= 
    \begin{pmatrix}
        E_\eta^{-1} \mathcal{K}_C + \mathrm{Pm} \mathcal{K}_\nu & \alpha^2 E_\eta^{-1} \Lambda \mathcal{K}_{L}(\hat{B}_0) \\ 
        \mathcal{K}_{I,u}(\hat{B}_0) & \mathcal{K}_{I,b}(\hat{u}_0) + \mathcal{K}_\eta
    \end{pmatrix}
    \begin{pmatrix} \hat{u} \\ \hat{b}' \end{pmatrix}
\end{aligned}\]
The only element different from the original system is the $\alpha^2$ factor in the Lorentz term. This can be fully absorbed by introducing $\Lambda' = \alpha^2 \Lambda$. 
Therefore, the result for the rescaled background magnetic field is the same as the un-rescaled version, with modified dimensionless numbers and rescaled perturbed fields
\begin{equation}
    E' = E, \quad E_\eta' = E_\eta, \quad \Lambda' = \alpha^2 \Lambda, \quad \hat{b}' = \alpha^{-1} \hat{b}.
\end{equation}
The (magnetic) Ekman numbers are left unchanged (as they should; neither the rotation time scale nor the diffusion time scales are altered), while the Elsasser number gains an additional $\alpha^2$ factor.
The effects on other dimensionless parameters follow directly from the conversion table above. For instance, the Lehnert and Lundquist numbers both gain an $\alpha$ factor
\[
    \mathrm{Le}' = \alpha \mathrm{Le},\quad 
    \mathrm{Lu}' = \alpha \mathrm{Lu}.
\]

\subsection{Dimensionless form of PG streamfunction equation}

Using the Alfvén time as the time scale, i.e. $\tau = L/V_A$ \parencite{holdenried-chernoff_long_2021}, the dimensionless momentum equation for inviscid flow takes the form
\[
    \frac{\partial \mathbf{u}}{\partial t} + \mathbf{u}\cdot \nabla \mathbf{u} + \frac{2}{\mathrm{Le}}\hat{\mathbf{z}}\times \mathbf{u} = -\nabla p + (\nabla\times\mathbf{B})\times \mathbf{B}
\]
and the streamfunction equation is
\[\begin{gathered}
    -2\nabla_e^2 \frac{\partial \psi}{\partial t} = \frac{dH}{ds} \left(\mathrm{Le}^{-1}\frac{4}{sH}\frac{\partial \psi}{\partial \phi} - \frac{2}{H}\frac{\partial}{\partial s}\frac{\partial \psi}{\partial t} - \frac{1}{sH}\frac{\partial^2}{\partial \phi^2}\frac{\partial \psi}{\partial t}\right) - \frac{dH}{ds}\left(2f_{e\phi} + \frac{1}{s}\frac{\partial \widetilde{f_\phi}}{\partial \phi}\right) + \hat{\mathbf{z}}\cdot \nabla\times \overline{\mathbf{f}_e} \\ 
    \left[\frac{\partial}{\partial s}\left(\frac{s}{H}\frac{\partial}{\partial s}\right) + \left(\frac{1}{sH} - \frac{1}{2H^2} \frac{dH}{ds}\right)\frac{\partial^2}{\partial \phi^2}\right] \frac{\partial \psi}{\partial t} = - \mathrm{Le}^{-1}\frac{2}{H^2}\frac{dH}{ds} \frac{\partial \psi}{\partial \phi} + \frac{dH}{ds} \left(\frac{s}{H} f_\phi^e + \frac{1}{2H}\frac{\partial \widetilde{f_z}}{\partial \phi}\right) - \frac{s}{2H}\hat{\mathbf{z}}\cdot \nabla\times \overline{\mathbf{f}_e}
\end{gathered}
\]
In constrast, \textcite{jackson_plesio-geostrophy_2020} uses the rotation time as the time scale, i.e. $\tau = \Omega^{-1}$, which is particularly useful when the magnetic field is absent. Now it is necessary to nondimensionalize the Lorentz force. While \textcite{jackson_plesio-geostrophy_2020} uses $\mathscr{B} = \sqrt{\rho_0 \mu_0} \Omega L$, meaning $\mathrm{Le} = 1$ in the paper, it lacks flexibility. Instead, using the Lehnert number, the Navier-Stokes equation for inviscid flow takes the form
\[
    \frac{\partial \mathbf{u}}{\partial t} + \mathbf{u}\cdot \nabla \mathbf{u} + 2\hat{\mathbf{z}}\times \mathbf{u} = -\nabla p + \mathrm{Le}^2(\nabla\times\mathbf{B})\times \mathbf{B}
\]
and the streamfunction equation is
\[\begin{gathered}
    -2\nabla_e^2 \frac{\partial \psi}{\partial t} = \frac{dH}{ds} \left(\frac{4}{sH}\frac{\partial \psi}{\partial \phi} - \frac{2}{H}\frac{\partial}{\partial s}\frac{\partial \psi}{\partial t} - \frac{1}{sH}\frac{\partial^2}{\partial \phi^2}\frac{\partial \psi}{\partial t}\right) - \mathrm{Le}^2 \left[\frac{dH}{ds}\left(2f_{e\phi} + \frac{1}{s}\frac{\partial \widetilde{f_\phi}}{\partial \phi}\right) + \hat{\mathbf{z}}\cdot \nabla\times \overline{\mathbf{f}_e}\right] \\ 
    \left[\frac{\partial}{\partial s}\left(\frac{s}{H}\frac{\partial}{\partial s}\right) + \left(\frac{1}{sH} - \frac{1}{2H^2} \frac{dH}{ds}\right)\frac{\partial^2}{\partial \phi^2}\right] \frac{\partial \psi}{\partial t} = - \frac{2}{H^2}\frac{dH}{ds} \frac{\partial \psi}{\partial \phi} + \mathrm{Le}^2 \left[\frac{dH}{ds} \left(\frac{s}{H} f_\phi^e + \frac{1}{2H}\frac{\partial \widetilde{f_z}}{\partial \phi}\right) - \frac{s}{2H}\hat{\mathbf{z}}\cdot \nabla\times \overline{\mathbf{f}_e}\right]
\end{gathered}
\]
The variables solved in two dimensionless forms can be easily converted to one another,
\[
    \mathbf{u}_\Omega = \frac{\mathbf{u}_A}{\Omega L} \frac{\mathscr{B}}{\sqrt{\rho_0 \mu_0}} = \mathrm{Le} \, \mathbf{u}_A, \qquad t_\Omega = \Omega \frac{\sqrt{\rho_0 \mu_0} L}{\mathscr{B}} t_A = \frac{t_A}{\mathrm{Le}},\qquad \mathbf{B}_\Omega = \mathbf{B}_A.
\]
Here the $A$ and $\Omega$ subscripts indicate dimensionless fields in the equations nondimensionalized using Alfvén wave velocity and rotation velocity, respectively. Finally, for the eigenvalue problem, the eigenvalues solved follow the following relation, inverse to $t$:
\[
    \omega_\Omega = \frac{\mathscr{B}}{\sqrt{\rho_0 \mu_0} \Omega L} \omega_A = \mathrm{Le} \, \omega_A.
\]
