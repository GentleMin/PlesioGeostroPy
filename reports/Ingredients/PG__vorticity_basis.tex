\section{Velocity components, vorticity and their bases}

I sometimes refer to the streamfunction equation as the vorticity equation. It may be an abuse of terminology but is somewhat justified as the streamfunction equation is derived from the axial vorticity equation.
Either way, the columnar ansatz dictates that the streamfunction has a one-to-one correspondence with the velocity components and vorticity. It follows that the bases we use for the streamfunction also induces the bases for the velocity components and vorticity.

We start by looking at the explicit expressions for velocity. The quasi-geostrophic ansatz gives
\begin{equation}
    u_s = \frac{1}{sH} \frac{\partial \psi}{\partial \phi}, \quad
    u_\phi = - \frac{1}{H} \frac{\partial \psi}{\partial s}, \quad 
    u_z = \frac{z}{sH^2} \frac{dH}{ds} \frac{\partial \psi}{\partial \phi}.
\end{equation}
The axial vorticity is expressed as
\begin{equation}
\begin{aligned}
    \zeta &= \hat{\mathbf{z}}\cdot \nabla\times \mathbf{u} = \hat{\mathbf{z}}\cdot \nabla_e \times \mathbf{u}_e = \hat{\mathbf{z}} \cdot \nabla_e \times \left(\frac{1}{H} \nabla_e \times \psi \hat{\mathbf{z}}\right) \\ 
    &= \hat{\mathbf{z}} \cdot \left(\nabla_e \frac{1}{H} \times (\nabla_e \psi \times \hat{\mathbf{z}}) + \frac{1}{H} \nabla_e \times (\nabla_e \psi \times \hat{\mathbf{z}})\right) \\ 
    &= \hat{\mathbf{z}} \cdot \left(-\frac{1}{H^2}\frac{dH}{ds} \hat{\mathbf{s}} \times \left(-\frac{\partial \psi}{\partial s} \hat{\bm{\phi}} + \frac{1}{s} \frac{\partial \psi}{\partial \phi} \hat{\mathbf{s}}\right) + \frac{1}{H} \nabla_e \times \left(-\frac{\partial \psi}{\partial s} \hat{\bm{\phi}} + \frac{1}{s} \frac{\partial \psi}{\partial \phi} \hat{\mathbf{s}}\right)\right) \\ 
    &= \hat{\mathbf{z}} \cdot \left(\frac{1}{H^2} \frac{dH}{ds} \frac{\partial \psi}{\partial s} \hat{\mathbf{z}} + \frac{\hat{\mathbf{z}}}{H} \left(- \frac{1}{s} \frac{\partial}{\partial s} \left(s \frac{\partial \psi}{\partial s}\right) - \frac{1}{s^2} \frac{\partial^2 \psi}{\partial \phi^2}\right)\right) \\ 
    &= - \frac{1}{s} \left(\frac{1}{H}\frac{\partial}{\partial s} \left(s \frac{\partial \psi}{\partial s}\right) - \frac{1}{H^2}\frac{dH}{ds} \frac{\partial \psi}{\partial s} + \frac{1}{sH} \frac{\partial^2 \psi}{\partial \phi^2}\right) \\ 
    \zeta &= - \frac{1}{s} \left[\frac{\partial}{\partial s} \left(\frac{s}{H} \frac{\partial}{\partial s}\right) + \frac{1}{sH}\frac{\partial^2}{\partial \phi^2}\right] \psi
\end{aligned}
\end{equation}
The axial vorticity proves just a scaled version of the first two terms of (\ref{eqn:streamfunction-eqn-fine}), which comes as no surprise, because the original streamfunction equation involves taking the axial vorticity.
Now we are ready to derive the bases for these quantities. We first consider the general Fourier expansion of the streamfunction
\[
    \psi(s, \phi, t) = \sum_{m,n} C_{\psi}^{mn}(t) \, \psi^{mn}(s) \, e^{im\phi}
\]
where $C^{mn}$ is the coefficient that varies with time, $\psi^{mn}(s)$ is the radial basis and $e^{im\phi}$ is of course the Fourier basis for the azimuth.
Consequently, the velocity field and the axial vorticity are expressed as
\begin{equation}
\begin{aligned}
    u_s &= \sum_{m,n} C_\psi^{mn}(t) \, \left[\frac{im}{sH} \psi^{mn}(s)\right] \, e^{im\phi}, \\ 
    u_\phi &= \sum_{m,n} C_\psi^{mn}(t) \, \left[-\frac{1}{H} \frac{d\psi^{mn}(s)}{ds}\right] \, e^{im\phi}, \\
    u_z &= \sum_{m,n} C_\psi^{mn}(t) \, \left[\frac{imz}{sH^2} \frac{dH}{ds} \psi^{mn}(s)\right] \, e^{im\phi}, \\
    \zeta &= \sum_{m,n} C_\psi^{mn}(t) \, \left[-\frac{1}{s}\frac{d}{ds} \left(\frac{s}{H} \frac{d\psi^{mn}(s)}{ds}\right) + \frac{m^2}{sH} \psi^{mn}(s)\right] \, e^{im\phi}.
\end{aligned}
\end{equation}
We see that these fields share the same coefficients and azimuthal bases (naturally, as we are always using Fourier series) as the streamfunction, but with modified radial bases, given in the square brackets. These bases will be referred to as $u_s^{mn}$, $u_\phi^{mn}$, $u_z^{mn}$ and $\zeta^{mn}$, respectively.
In a full sphere where $H=\sqrt{1 - s^2}$, we know that the hydrodynamic system has the inertial modes described by streamfunction
\[
    \psi_\mathrm{inertial}^{nm} = s^{|m|} H^3 P_n^{(\frac{3}{2}, |m|)}(2s^2 - 1) \, e^{im\phi}.
\]
The radial bases here form a complete orthogonal basis in the inner product space of analytic functions with prefactors $s^{|m|} H^3$ within interval $[0, 1]$ with weight function $1$. The prefactor $s^{|m|}$ is necessary for the scalar to be regular (see chapter \ref{chap:regularity}), and the prefactor $H^3$ is necessary for the underlying velocity field to be regular. This has been since used as the basis for the spectral expansion of the streamfunction. Following this basis, the corresponding velocity radial basis takes the form
\begin{equation}
\begin{aligned}
    u_s^{mn}(s) &= \frac{im}{sH} \psi^{mn}(s) = im s^{|m|-1} H^2 P_n^{(\frac{3}{2}, |m|)}(2s^2 - 1), \\ 
    u_\phi^{mn}(s) &= - \frac{1}{H} \frac{d\psi^{mn}(s)}{ds} = - \left(s^{|m|}H^2 \frac{d}{ds} + |m|s^{|m|-1} H^2 + 3s^{|m|}H \frac{dH}{ds}\right) P_n^{(\frac{3}{2},|m|)}(2s^2 - 1) \\ 
    &= -s^{|m|-1} \left(sH^2 \frac{d}{ds} + |m| H^2 - 3s^2\right) P_n^{(\frac{3}{2}, |m|)}(2s^2 -1) \\ 
    &= s^{|m|-1} \left(3s^2 - |m|H^2\right) P_n^{(\frac{3}{2}, |m|)}(2s^2 - 1) - s^{|m|+1}H^2\left(n + |m| + \frac{5}{2}\right) P_{n-1}^{\frac{5}{2},|m|+1}(2s^2 - 1), \\ 
    u_z^{mn}(s) &= \frac{imz}{sH^2} \frac{dH}{ds} \psi^{mn}(s) = - imz s^{|m|} P_n^{(\frac{3}{2}, |m|)}(2s^2 - 1).
\end{aligned}
\end{equation}
The situation with the axial vorticity is slightly more complicated. Instead of plugging in directly the expression, we observe that the radial streamfunction basis is the solution to the following equation,
\[
    \left[\frac{d}{ds} \left(\frac{s}{H} \frac{d}{ds}\right) - \frac{m^2}{sH} - \frac{m^2 s}{2H^3}\right] \psi_\mathrm{inertial}^{mn}(s) = \frac{2s}{H^3} \frac{m}{\omega_\psi^{mn}} \psi_\mathrm{inertial}^{mn}(s)
\]
where $\omega_\psi^{mn}$ is the eigenfrequency of the inertial mode in the PG model. This gives us the relation,
\begin{equation}
\begin{aligned}
    \zeta^{mn}(s) &= -\frac{1}{s} \left[\frac{d}{ds} \left(\frac{s}{H} \frac{d}{ds}\right) - \frac{m^2}{sH}\right] \psi_\mathrm{inertial}^{mn}(s) \\
    &= \left(-2\frac{m}{\omega_\psi^{mn}} - \frac{m^2}{2}\right) \frac{1}{H^3} \psi_\mathrm{inertial}^{mn}(s) \\ 
    &= \left(2(n+1)(2n+2m+3) + m\right) \frac{1}{H^3} \psi_\mathrm{inertial}^{mn}(s) \\ 
    &= \left(2(n+1)(2n+2m+3) + m\right) s^{|m|} P_n^{(\frac{3}{2}, |m|)}(2s^2 - 1).
\end{aligned}
\end{equation}
The behaviour of these bases will be illustrated in two approaches. First, we can analyze the asymptotic behaviour near the equator by plugging in $s^2 = 1-H^2$ and taking $H$ to approach zero. This gives
\[
\begin{aligned}
    u_s^{mn}(s) &= im H^2 P_n^{(\frac{3}{2},|m|)} (1-2H^2) + O(H^4),\\ 
    u_\phi^{mn}(s) &= 3 P_n^{(\frac{3}{2},|m|)} (1-2H^2) + O(H^2),\\
    u_z^{mn}(s) &= -imz P_n^{(\frac{3}{2},|m|)} (1-2H^2),\\
    \zeta^{mn}(s) &= \left(2(n+1)(2n+2m+3) + m\right) P_n^{(\frac{3}{2},|m|)} (1-2H^2) + O(H^2),
\end{aligned}
\]
As seen from here, the basis for $u_s$ automatically vanishes at $s=1$ (as it should). This is not the case with azimuthal velocity or vorticity, whose limits at $H\rightarrow 0$ are given by
\[\begin{aligned}
    u_\phi^{mn}(s)|_{s\rightarrow 1} &= 3 P_n^{(\frac{3}{2}, |m|)}(1) = 3 \begin{pmatrix} n + \frac{3}{2} \\ n \end{pmatrix},\\
    \zeta^{mn}(s)|_{s\rightarrow 1} &= \left(2(n+1)(2n+2m+3) + m\right) P_n^{(\frac{3}{2}, |m|)}(1) \\ 
    &= \left(2(n+1)(2n+2m+3) + m\right) \begin{pmatrix} n + \frac{3}{2} \\ n \end{pmatrix}
\end{aligned}\]
respectively. Not only do they not vanish at $s=1$, but these values grow algebraically with degree $n$ as one goes to higher degrees of Jacobi polynomials. (Note: use the asymptotic formula $\begin{psmallmatrix} n+\alpha \\ n \end{psmallmatrix} \sim n^\alpha/\Gamma(\alpha + 1)$, see wikipedia page for binomial coefficients) 
Considering that Jacobi polynomials also tend to have large values at boundaries, the basis functions for the azimuthal velocity and vorticity are increasingly concentrated at the boundary rather than the bulk with increasing degrees.

As a second approach, we present here the first, third, fifth and tenth radial basis function for the streamfunction, the radial and azimuthal velocity, and the vorticity, at azimuthal wavenumber $m=3$ (Fig.\ref{fig:rad-basis}). 
\begin{figure}[htbp]
    \centering
    \includegraphics[width=\linewidth]{../../out/imgs/basis_velocity_vorticity.pdf}
    \caption{The radial bases for streamfunction, equatorial velocity and axial vorticity.}
    \label{fig:rad-basis}
\end{figure}
This illustrates the large amplitudes of $u_\phi^{mn}$ and $\zeta^{mn}$ near the boundary.
As a result, eigenmodes that consist of only one of this basis (e.g. inertial modes, magnetic eigenmodes with Malkus background fields, etc.) will inherit this feature that the azimuthal velocity at the boundary is much larger than that in the bulk or the radial velocity anywhere in the system, and the vorticity is concentrated at the boundary.

\clearpage
