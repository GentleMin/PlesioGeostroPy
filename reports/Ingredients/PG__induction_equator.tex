\section{Induction equation in the equatorial plane}

The induction equations in the equatorial plane are obtained by taking the original induction equation at the equatorial plane, i.e. $z=0$.
Therefore, it is useful to first introduce this \textit{equator sampling operator},
\[
    \mathcal{S}^e: V(\mathbb{R}^3) \mapsto V(\mathbb{R}^2)
\]
where $V(\mathbb{R}^3)$ is some vector space of functions defined in 3-D space, and $V(\mathbb{R}^2)$ is some vector space of functions defined on 2-D plane.
The sampling operator is simply defined as
\[
    \mathcal{S}^e f(\mathbf{r}) = f|_{z=0} = f(\mathbf{r}_e, z=0) = f^e(\mathbf{r}_e).
\]
Here the superscript $e$ is used to mark a quantity that is evaluated in the equatorial plane. The subscript $e$ is reserved to denote equatorial components of a vector field.
% Although the expression above only gives the case where $f$ is a function in $x,y,z$, the same process follows regardless of coordinate systems. For functions in cylindrical coordinates, there is no difference.
Operator $\mathcal{S}^e$ is linear (naturally),
\[
    \mathcal{S}^e \left(\alpha f + \beta g\right) = \alpha \mathcal{S}^e f + \beta \mathcal{S}^e g = \alpha f^e + \beta g^e
\]
where $f$ and $g$ are functions while $\alpha$ and $\beta$ are constants. It is also distributive,
\[
    \mathcal{S}^e \left(f g\right) = \left(\mathcal{S}^e f\right) \left(\mathcal{S}^e g\right) = f^e g^e.
\]
Last but not least, it commutes with differential operators in the equatorial plane, as well as functions defined using only equatorial components
\[\begin{aligned}
    \mathcal{S}^e \left(f(\mathbf{r}_e) g(\mathbf{r})\right) &= f(\mathbf{r}_e) \mathcal{S}^e g = f(\mathbf{r}_e) g^e(\mathbf{r}_e), \\ 
    \mathcal{S}^e \left(\nabla_e g(\mathbf{r})\right) &= \nabla_e \left(\mathcal{S}^e g\right) = \nabla_e g^e(\mathbf{r}_e).
\end{aligned}
\]
The properties above are intuitive results, and I shall omit the proof here. Rigorous proof is possible by simply using the definition, sometimes combined with Cartesian coordinates.
Now let us return to the induction equation. The original equation takes the form,
\[
    \frac{\partial \mathbf{B}}{\partial t} = \nabla \times (\mathbf{u} \times \mathbf{B}) = \left(\mathbf{B}\cdot \nabla\right) \mathbf{u} - \left(\mathbf{u} \cdot \nabla\right) \mathbf{B}.
\]
To make use of the listed properties, we split the vectors as well as the nabla operator into equatorial and vertical components, i.e. $\mathbf{u} = \mathbf{u}_e + u_z \hat{\mathbf{z}}$, $\mathbf{B} = \mathbf{B}_e + B_z \hat{\mathbf{z}}$ and $\nabla = \nabla_e + \hat{\mathbf{z}} \partial_z$. The equation reads
\begin{equation}\label{eqn:induction-eqn-equatorial-vertical-split}
\begin{aligned}
    \frac{\partial \mathbf{B}}{\partial t} &= \left(\mathbf{B}_e\cdot \nabla_e + B_z \partial_z\right) \left(\mathbf{u}_e + u_z \hat{\mathbf{z}}\right) - \left(\mathbf{u}_e \cdot \nabla_e + u_z \partial_z\right) \mathbf{B} \\ 
    &= \mathbf{B}_e\cdot \nabla_e \mathbf{u}_e + B_z \partial_z \mathbf{u}_e - \mathbf{u}_e \cdot \nabla_e \mathbf{B} - u_z \partial_z \mathbf{B} + \hat{\mathbf{z}} \left(\mathbf{B}_e\cdot \nabla_e u_z + B_z \partial_z u_z \right) \\ 
    &= \mathbf{B}_e\cdot \nabla_e \mathbf{u}_e - \mathbf{u}_e \cdot \nabla_e \mathbf{B} - u_z \partial_z \mathbf{B} + \hat{\mathbf{z}} \left(\mathbf{B}_e\cdot \nabla_e u_z + B_z \partial_z u_z \right)
\end{aligned}
\end{equation}
where the column ansatz is used in the last step to remove the term containing $\partial_z \mathbf{u}_e$.
To obtain the evolution equation for $\mathbf{B}(z=0) = \mathbf{B}^e$, we apply the sampling operator to eq.(\ref{eqn:induction-eqn-equatorial-vertical-split}),
\begin{equation}
    \frac{\partial \mathbf{B}^e}{\partial t} = \mathcal{S}^e \frac{\partial \mathbf{B}}{\partial t} = \left(\mathbf{B}_e^e \cdot \nabla_e\right) \mathbf{u}_e - \left(\mathbf{u}_e\cdot \nabla_e\right) \mathbf{B}^e + \hat{\mathbf{z}} B_z^e \left(\partial_z u_z\right)^e.
\end{equation}
Here we used the property $u_z^e = 0$, i.e. the axial velocity is zero at the equatorial plane, again dictated by the columnar ansatz.
To obtain the evolution equation for $\partial_z\mathbf{B}_e|_{(z=0)} = \mathbf{B}_{e,z}^e$, we apply $\mathcal{S}^e \partial_z$ to eq.(\ref{eqn:induction-eqn-equatorial-vertical-split}), and take only the equatorial components,
\begin{equation}
    \frac{\partial \mathbf{B}_{e,z}^e}{\partial t} = \mathcal{S}^e \partial_z \frac{\partial \mathbf{B}_e}{\partial t} = \left(\mathbf{B}_{e,z}^e \cdot \nabla_e\right) \mathbf{u}_e - \left(\mathbf{u}_e\cdot \nabla_e\right) \mathbf{B}_{e,z}^e - \mathbf{B}_{e,z}^e \left(\partial_z u_z\right)^e.
\end{equation}
