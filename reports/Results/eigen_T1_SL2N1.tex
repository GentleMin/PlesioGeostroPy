\section{Ideal eigenmodes under T1-SL2N1, a mixed toroidal-poloidal equatorially symmetric magnetic field}

The background field, which we call T1-SL2N1, is defined as

\begin{figure}[h]
    \centering
    \includegraphics[width=\linewidth]{../../out/eigen/T1_SL2N1_std/bg_B_T1_SL2N1_uscale.png}
    \caption{Background field T1-SL2N1}
\end{figure}

This is a background field with purely equatorially symmetric property (i.e. has mirror symmetry \wrt the equatorial plane, also sometimes referred to as "quadrupolar" parity). Background fields T1 and SL2N1 have this symmetry individually, but instead of having either purely poloidal or purely toroidal component, the current background field is a mixture of both poloidal and toroidal fields.

Linearized equations omitted.

\subsection{System spectrum}

\begin{figure}[h]
    \centering
    \includegraphics[width=\linewidth]{../../out/eigen/T1_SL2N1_std/Spectrum_m3_Le1e-4_ideal_cEW_raw_Original-Canonical-Reduced.pdf}
    \caption{Spectrum of three different spectral methods at the same resolution}
\end{figure}

\begin{figure}[h]
    \centering
    \includegraphics[width=\linewidth]{../../out/eigen/T1_SL2N1_std/Canonical/Spectrum_m3_Le1e-4_ideal_cEW_raw.pdf}
\end{figure}


\subsection{Selected eigenmodes}


\clearpage

