\section{Ideal eigenmodes under poloidal dipolar background field}

The background field is given by
\[
    \mathbf{B}^0 = -6sz \hat{\mathbf{s}} + (12s^2 + 6z^2 - 10) \hat{\mathbf{z}}
\]
which translates to the following background PG quantities
\begin{equation}
\begin{aligned}
    \overline{M_{ss}}^0 &= 24 H^{3} s^{2} \\ 
    \widetilde{M_{sz}}^0 &= - 18 H^{4} s + 2 H^{2} \left(- 36 s^{3} + 30 s\right) \\
    \widetilde{zM_{ss}}^0 &= 18 H^{4} s^{2} \\
    B_{z}^{0e} &= 12 s^{2} - 10 \\
    B_{s, z}^{0e} &= - 6 s \\
    B_s^{0+} &= - 6 H s \\
    B_z^{0+} &= 6 s^{2} - 4 \\
    B_s^{0-} &= 6 H s \\
    B_z^{0-} &= 6 s^{2} - 4 \\
    \Psi^{0} &= \overline{M_{\phi\phi}}^0 = \overline{M_{s\phi}}^0 = \widetilde{M_{\phi z}}^0 = \widetilde{zM_{\phi\phi}}^0 = \widetilde{zM_{s\phi}}^0 = 0 \\ 
    B_{s}^{0e} &= B_{\phi}^{0e} = B_{\phi, z}^{0e} = B_\phi^{0+} = B_\phi^{0-} = 0.
\end{aligned}
\end{equation}

\subsection{Linearized equations}

As always, the most complicated part is the streamfunction equation. Under the poloidal dipolar background field, it takes the form,
\begin{equation}\label{eqn:eigen-pgpsi-poloidal-dipole}
\begin{aligned}
    i\widetilde{\omega} \left[\frac{d}{d s}\left(\frac{s}{H}\frac{d}{d s}\right) - m^2 \left(\frac{1}{sH} + \frac{s}{2H^3}\right)\right] \psi &= \frac{2 i m s}{\mathrm{Le} H^{3}} \psi \\
    &\mkern-240mu + \frac{i m}{2 H} \frac{d\overline{m_{ss}}}{d s} + \frac{i m}{2 H s} \overline{m_{ss}} - \frac{i m}{2 H} \frac{d \overline{m_{\phi\phi}}}{d s} - \frac{i m}{2 H s} \overline{m_{\phi\phi}} - \frac{s}{2 H} \frac{d^{2}\overline{m_{s\phi}}}{d s^{2}} - \frac{3}{2 H} \frac{d\overline{m_{s\phi}}}{d s} - \frac{m^{2}}{2 sH} \overline{m_{s\phi}} \\
    &\mkern-240mu - \frac{i m s}{2 H^{2}} \frac{d\widetilde{m_{sz}}}{d s} - \frac{i m}{2 H^{2}}\widetilde{m_{sz}} + \frac{m^{2}}{2 H^{2}} \widetilde{m_{\phi z}} + 4 i ms\frac{6 s^{2} - 5}{H^{2}} b_{z}^e - 2s^2\frac{6 s^{2} - 5}{H^{2}} b_{\phi, z}^e \\
    &\mkern-240mu + \frac{2 s}{H} \left(\frac{db^{+}_{\phi}}{d s} - \frac{db^{-}_{\phi}}{d s}\right) + \frac{2}{H} \left(b^{+}_{\phi} - b^{-}_{\phi}\right) + \frac{i m s}{H^{2}} \left(b^{+}_{z} + b^{-}_{z}\right) + \frac{i m \left(s^2 - 2\right)}{H^{3}} \left(b^{+}_{s} - b^{-}_{s}\right).
\end{aligned}
\end{equation}
The magnetic quantities are described by the induction equations in the PG model, which are given by
\begin{equation}\label{eqn:eigen-pgmag-poloidal-dipole}
\begin{aligned}
    i \widetilde{\omega} \overline{m_{ss}} &= i48m \left[ sH^2 \frac{d\psi}{d s} + \left(4 s^{2} - 2\right) \psi\right] \\ 
    % i \widetilde{\omega} \overline{m_{s\phi}} &= - 24 H^{2} s^{2} \frac{d^{2}}{d s^{2}} \psi + \left(24 H^{2} s - 24 s^{3}\right) \frac{d}{d s} \psi \\
    i \widetilde{\omega} \overline{m_{s\phi}} &= 24 \left[-s^2 H^{2} \frac{d^{2} \psi}{d s^{2}} + s\left(1-2s^2\right) \frac{d\psi}{d s}\right] \\
    % i \widetilde{\omega} \widetilde{m_{sz}} &= \left(- \frac{324 i m s^{3}}{H} + \frac{342 i m s}{H} - \frac{84 i m}{H s}\right) \psi + \left(\frac{72 i m s^{4}}{H} - \frac{114 i m s^{2}}{H} + \frac{42 i m}{H}\right) \frac{d}{d s} \psi \\
    i \widetilde{\omega} \widetilde{m_{sz}} &= i6m \left[\frac{12s^4 - 19s^2 + 7}{H} \frac{d\psi}{d s} - \frac{54s^4 - 57s^2 + 14}{sH} \psi\right] \\
    % i \widetilde{\omega} \widetilde{m_{\phi z}} &= \left(54 H s^{3} - 42 H s\right) \frac{d^{2}}{d s^{2}} \psi + \left(- 54 H s^{2} + 42 H + \frac{54 s^{4}}{H} - \frac{42 s^{2}}{H}\right) \frac{d}{d s} \psi \\
    i \widetilde{\omega} \widetilde{m_{\phi z}} &= 6 \left[sH \left(9 s^{2} - 7\right) \frac{d^{2}\psi}{d s^{2}} + \frac{1}{H}\left(2s^2 - 1\right)\left(9s^2 - 7\right) \frac{d\psi}{d s}\right] \\
    % i \widetilde{\omega} \widetilde{zm_{ss}} &= 36 i H^{3} m s \frac{d}{d s} \psi + \left(- 36 i H^{3} m + 108 i H m s^{2} - 36 i H m\right) \psi \\
    i \widetilde{\omega} \widetilde{zm_{ss}} &= i36 m \left[s H^{3} \frac{d\psi}{d s} + 2H \left(2 s^{2} - 1 \right) \psi\right] \\
    % i \widetilde{\omega} \widetilde{zm_{s\phi}} &= - 18 H^{3} s^{2} \frac{d^{2}}{d s^{2}} \psi + \left(18 H^{3} s - 18 H s^{3}\right) \frac{d}{d s} \psi \\
    i \widetilde{\omega} \widetilde{zm_{s\phi}} &= 18 \left[- s^2 H^{3} \frac{d^{2}\psi}{d s^{2}} + sH \left(1 - 2s^2\right) \frac{d \psi}{d s} \right] \\
    % i \widetilde{\omega} b_{z}^e &= \left(\frac{12 i m s^{2}}{H^{3}} - \frac{14 i m}{H^{3}}\right) \psi \\
    i \widetilde{\omega} b_{z}^e &= i2m \frac{6s^2 - 7}{H^3} \psi \\
    % i \widetilde{\omega} b_{s, z}^e &= \left(\frac{12 i m}{H s} - \frac{12 i m s}{H^{3}}\right) \psi - \frac{6 i m \frac{d}{d s} \psi}{H} \\
    i \widetilde{\omega} b_{s, z}^e &= i6m \left[- \frac{1}{H}\frac{d\psi}{d s} + \frac{2(1 - 2s^2)}{sH^3} \psi\right] \\
    % i \widetilde{\omega} b_{\phi, z}^e &= \left(- \frac{6}{H} + \frac{6 s^{2}}{H^{3}}\right) \frac{d}{d s} \psi + \frac{6 s \frac{d^{2}}{d s^{2}} \psi}{H} \\
    i \widetilde{\omega} b_{\phi, z}^e &= 6\left[\frac{s}{H}\frac{d^{2}\psi}{d s^{2}} + \frac{2s^2 - 1}{H^3} \frac{d\psi}{d s} \right] \\
    \overline{m_{\phi\phi}} &= \widetilde{zm_{\phi\phi}} = b_{s}^e = b_{\phi}^e = 0
\end{aligned}
\end{equation}
All the PG induction equations are validated against and are exactly the same as in \textcite{holdenried-chernoff_long_2021}.
In contrast, the boundary induction equations are given as follows
\begin{equation}\label{eqn:eigen-bound-poloidal-dipole}
\begin{aligned}
    i \widetilde{\omega} b^{+}_{s} &= - i \widetilde{\omega} b^{-}_{s} = i 6m \left[-\frac{d\psi}{d s} + \frac{2(1 - 2s^2)}{sH^2} \psi\right], \\
    % i \widetilde{\omega} b^{+}_{\phi} &= -i \widetilde{\omega} b^{-}_{\phi} = 6 s \frac{d^{2}}{d s^{2}} \psi + \left(-6 + \frac{6 s^{2}}{H^{2}}\right) \frac{d}{d s} \psi \\
    i \widetilde{\omega} b^{+}_{\phi} &= -i \widetilde{\omega} b^{-}_{\phi} = 6 \left[s \frac{d^{2}\psi}{d s^{2}} + \frac{2s^2 - 1}{H^2} \frac{d\psi}{d s} \right], \\ 
    i \widetilde{\omega} b^{+}_{z} &= i \widetilde{\omega} b^{-}_{z} = i2m \left[\frac{3 s}{H} \frac{d\psi}{d s} + \frac{4(3s^2 - 1)}{H^3} \psi \right].
\end{aligned}
\end{equation}
While the meridional components $b_s^{\pm}$ and $b_z^{\pm}$ share the same expression as in \textcite{holdenried-chernoff_long_2021}, the azimuthal components $b_\phi^{\pm}$ are not mentioned in the previous work.
It is unclear what argument was made to ignore this boundary term, but the calculation here seems to suggest that this component is however not trivial.
Is this used or not in the previously obtained results? This is a question that only the original author can answer.

We see that contrary to the boundary fields for Malkus field, the boundary fields for the poloidal dipolar background field has another parity.
The equatorial components $b_s$ and $b_\phi$ are opposite at the upper and the lower boundary, while the vertical component $b_z$ matches.
This means that the boundary terms in (\ref{eqn:eigen-pgpsi-poloidal-dipole}) do not cancel out like they do in (\ref{eqn:eigen-pg-malkus}), and thus boundary terms do matter.


\subsection{Standard ODE form}

To check the regularity of the system, we form the standard ODE of $\psi$ as a function of $s$ by merging the PG equations and boundary induction equations into one single ODE (\ref{eqn:ode-poloidal-dipole}).
The standard form is a 4-th order system in $\psi$. The singularities of the equations are $s=0$, i.e. at the axis, and $H=0$ ($s=1$), i.e. at the equator. The singularities of the coefficients are analyzed below.
\begin{equation}\label{eqn:ode-poloidal-dipole}
\begin{aligned}
    &\frac{d^{4}\psi}{d s^{4}} + \frac{8 - 9 s^{2}}{sH^{2}} \frac{d^{3}\psi}{d s^{3}} \\
    +& \left(\frac{\omega^{2}}{12 H^{2} s^{2}} + \frac{- m^{2} \left(s^{4} - 5 s^{2} + 4\right) - 16 s^{2} + 20}{4 s^{2} H^{4}}\right) \frac{d^{2}\psi}{d s^{2}} \\
    +& \left(\frac{\omega^{2}}{12 s^{3} H^{4}} + \frac{- m^{2} \left(s^{6} - 2 s^{4} - 2 s^{2} + 3\right) + 18 s^{6} - 51 s^{4} + 41 s^{2} - 5}{s^3 H^{6}}\right) \frac{d\psi}{d s} \\
    +& \left(\frac{\omega^{2} m^{2} \left(s^2 - 2\right)}{24 s^{4} H^{4}} - \frac{1}{\mathrm{Le}} \frac{\omega m}{6 s^{2} H^{4}} - \frac{m^{2} \cdot \left(30 s^{6} - 127 s^{4} + 156 s^{2} - 32\right)}{4 s^{4} H^{6}}\right) \psi = 0
\end{aligned}
\end{equation}
% \begin{itemize}
%     \item The 3-rd order term has a simple pole at $s=0$ and a simple pole at $s=1$.
%     \item The 2-nd order term has a 2-nd order pole at $s=0$ and a 2-nd order pole at $s=1$.
%     \item The 1-st order term has a 3-rd order pole at $s=0$ and a 3-rd order pole at $s=1$.
%     \item The 0-th order term has a 4-th order pole at $s=0$ and a 3-rd order pole at $s=1$.
% \end{itemize}

The 3-rd order term has a simple pole at $s=0$ and a simple pole at $s=1$;

The 2-nd order term has a 2-nd order pole at $s=0$ and a 2-nd order pole at $s=1$;

The 1-st order term has a 3-rd order pole at $s=0$ and a 3-rd order pole at $s=1$;

The 0-th order term has a 4-th order pole at $s=0$ and a 3-rd order pole at $s=1$.

The overall conclusion is that the singularities of the equation are both regular singularities, so regular solutions should exist.
However, singularity aside, these coefficients differ from the ones as provided in \textcite{holdenried-chernoff_long_2021}. Some coefficients are the same, but not all. Recalling that in the previous work, the $b_\phi^{\pm}$ may be ignored, I tried to remove these equations before assembling the 2-nd order ODE. The result is as follows
\begin{equation}
\begin{aligned}
    &\frac{d^{4}\psi}{d s^{4}} + \frac{6 - 9 s^{2}}{sH^{2}} \frac{d^{3}\psi}{d s^{3}} \\
    +& \left(\frac{\omega^{2}}{12 s^{2} H^{2}} + \frac{m^{2} \left(- s^{4} + 5 s^{2} - 4\right) - 16 s^{2} + 12}{4 s^{2} H^{4}}\right) \frac{d^{2}\psi}{d s^{2}} \\
    +& \left(\frac{\omega^{2}}{12 s^{3} H^{4}} + \frac{- m^{2} \left(s^{6} - 2 s^{4} - 2 s^{2} + 3\right) + 18 s^{6} - 47 s^{4} + 31 s^{2} - 3}{s^{3} H^{6}}\right) \frac{d\psi}{d s} \\
    +& \left(\frac{\omega^{2} m^{2} \left(s^2 - 2\right)}{24 s^{4} H^{4}} - \frac{1}{\mathrm{Le}} \frac{\omega m}{6 s^{2} H^{4}} - \frac{m^{2} \cdot \left(30 s^{6} - 127 s^{4} + 156 s^{2} - 32\right)}{4 s^{4} H^{6}}\right) \psi = 0
\end{aligned}
\end{equation}
Now the first three coefficients all match with \textcite{holdenried-chernoff_long_2021}. However, the coefficients of the lowest two order terms still do not match. The only difference, as it seems, lies in the terms containing $m^2$ in these coefficients.
It does not seem to be possible to tell the right from wrong here, so a thorough symbolic re-analysis seems necessary.

One notable feature that might aid debugging is that according to the equations presented here, there aren't even terms that can give $s^7$ on the numerator. However, both discrepant coefficients contain $s^7m^2$ terms in \textcite{holdenried-chernoff_long_2021}. "Dimension" analysis of the streamfunction equation (\ref{eqn:eigen-pgpsi-poloidal-dipole}) seems to indicate that such a high degree is not possible. If something is wrong with the derivation, the mistake may have already occurred at eq.(\ref{eqn:eigen-pgpsi-poloidal-dipole}). Understanding how this $s^7$ term comes about might be a key to finding out potential errors.


\subsection{System spectrum}

\begin{figure}[htbp]
    \centering
    \includegraphics[width=.9\linewidth]{../../out/eigen/Poloidal_Dipole/Reduced/Spectrum_m3_Le-1.414e-4.pdf}
    \caption{Spectrum for $m=3$ modes.}
\end{figure}

\subsection{Selected eigenmodes}

\clearpage
