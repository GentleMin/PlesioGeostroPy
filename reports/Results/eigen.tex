\chapter{Solutions to the eigenvalue problems}

I present here the solutions to the eigenvalue problems in the PG model.
The full PG model is a dynamical system described by system of PDEs
\[
    \mathcal{M}_i \frac{\partial x_i}{\partial t} = \mathcal{F}_i (x_i, \cdots x_N), \qquad i = 1, 2\cdots N.
\]
$x_i$ are the dynamical variables in the PG model, $\Psi$, $\overline{M_{ss}}$, $\overline{M_{\phi\phi}}$, etc. If the boundary magnetic field is described by $B_r$, then $N=15$; alternatively, if the boundary magnetic field is described by $B_s^{\pm}$, $B_\phi^{\pm}$ and $B_z^{\pm}$, then $N=20$.
The former provides a closed system under certain boundary conditions, but requires a cylindrical-to-spherical transform. On the other hand, the latter is intrinsically in cylindrical coordinates, but is not closed in the nonlinear case. However, in eigenvalue problems with static background flow, the latter formulation is closed. For details, refer to the Ingredients document.
Introducing
\[
    \mathcal{M} = \begin{pmatrix}
        \mathcal{M}_1 & & \\ 
        & \ddots & \\
        & & \mathcal{M}_N
    \end{pmatrix},\qquad 
    \mathcal{F} = \begin{pmatrix} \mathcal{F}_1 \\ \vdots \\ \mathcal{F}_N \end{pmatrix}, \qquad 
    \mathbf{x} = \begin{pmatrix} x_1 \\ \vdots \\ x_N \end{pmatrix},
\]
we can formally write the system as
\[
    \mathcal{M} \frac{\partial \mathbf{x}}{\partial t} = \mathcal{F}(\mathbf{x}).
\]
If we consider perturbations near a time-invariant background field denoted as $\mathbf{x}^0$, the perturbed quantities follow the linearized equations
\[
    \mathcal{M} \frac{\partial \mathbf{x}}{\partial t} = D\mathcal{F}(\mathbf{x}^0) \, \mathbf{x} = \frac{\partial \mathcal{F}(\mathbf{x}^0)}{\partial \mathbf{x}^0} \, \mathbf{x} = \mathcal{K}(\mathbf{x}^0) \, \mathbf{x}.
\]
This linear, autonomous dynamical system admits general solutions in the form of time-harmonic functions $\mathbf{x}(t) = \mathbf{x}(0) e^{\lambda t} = \mathbf{x}^0 e^{i\widetilde{\omega}t}$.
The purpose of this chapter is therefore to present the solution of eigenvalues $\lambda$ and eigenmodes $\mathbf{x}(t=0)$ to the eigenvalue problem in the form of
\begin{equation}\label{eqn:eigenproblem}
    \lambda \mathcal{M} \mathbf{x} = i \widetilde{\omega} \mathcal{M} \mathbf{x} = \mathcal{K}(\mathbf{x}^0) \, \mathbf{x}.
\end{equation}
Unless otherwise specified, throughout this chapter the term \textit{eigenvalue} refers to $\lambda = i\widetilde{\omega} \in \mathbb{C}$ defined in eq.(\ref{eqn:eigenproblem}). $\sigma = \mathrm{Re}[\lambda]$ gives the exponential growth / decay rate, while $\omega = \mathrm{Im}[\lambda]$ gives the angular frequency of temporal oscillations (\textit{eigenfrequency}).
For systems free of both viscous and magnetic diffusion, physics dictates that $\mathrm{Re}[\lambda] = 0$ or $\widetilde{\omega} = \omega$, except for small numerical errors. Hereinafter such systems are described as \textit{ideal}.

I shall present the solutions to the eigenvalue problems under several background fields $\mathbf{x}^0$. In each case, I shall provide the solved system, simplified equation, spectrum of the system, as well as selected eigenmodes.
In this chapter, \textit{the spectrum of a system or matrix} refers to the set of eigenvalues, whereas \textit{the spectrum of an eigenmode} refers to the composition of the mode in terms of basis functions.

\clearpage

\section{Inviscid hydrodynamic eigenmodes}

We start by considering the eigenmodes in absence of magnetic fields in the inviscid limit.
From the ideal PG equations, it means that the system is linearized around a background state where both the velocity and the magnetic fields are zero.

\subsection{Linearized equations}

The PG system in purely invscid hydrodynamic case comprises only of the streamfunction equation,
\begin{equation}
    \left[\frac{\partial}{\partial s}\left(\frac{s}{H}\frac{\partial}{\partial s}\right) + \left(\frac{1}{sH} + \frac{s}{2H^3}\right)\frac{\partial^2}{\partial \phi^2}\right] \frac{\partial \psi}{\partial t} = \frac{2 s}{H^{3}} \frac{\partial \psi}{\partial \phi}
\end{equation}
while all magnetic quantities vanish. There is no difference between the PG equation, the transformed equation, or the reduced dimensional formulation, as only the streamfunction is relevant.
Note for the hydrodynamic case, the rotation timescale $\tau = \Omega^{-1}$ is used.

\subsection{Standard ODE form}

Using the Fourier ansatz $\psi = \psi^m(s) e^{i \widetilde{\omega} t + im\phi} = \psi^m(s) e^{\lambda t + im \phi}$, the streamfunction equation can be written as an ODE in cylindrical radius $s$,
\begin{equation}\label{eqn:ode-hydro}
\begin{aligned}
    \lambda \left[\frac{d}{d s}\left(\frac{s}{H}\frac{d}{d s}\right) - m^2 \left(\frac{1}{sH} + \frac{s}{2H^3}\right)\right] \psi^m &= \frac{2 s}{H^{3}} im \psi^m \\ 
    \widetilde{\omega} \left[\frac{d}{d s}\left(\frac{s}{H}\frac{d}{d s}\right) - m^2 \left(\frac{1}{sH} + \frac{s}{2H^3}\right)\right] \psi^m &= \frac{2 s}{H^{3}} m \psi^m \\
    \left[\frac{d}{d s}\left(\frac{s}{H}\frac{d}{d s}\right) - m^2 \left(\frac{1}{sH} + \frac{s}{2H^3}\right)\right] \psi^m &= \frac{2 s}{H^{3}} \frac{m}{\widetilde{\omega}} \psi^m,
\end{aligned}
\end{equation}
which can also be cast into the standard form,
\begin{equation}
    \frac{d^{2}}{d s^{2}} \psi^{m} + \frac{1}{s H^{2}}\frac{d}{d s} \psi^{m} - \left(\frac{m^{2} \left(H^{2} + 1\right)}{2 s^{2} H^{2}} + \frac{2 m}{\widetilde{\omega} H^{2}}\right) \psi^{m} = 0.
\end{equation}
All coefficients in the standard form are rational forms of cylindrical radius $s$. The poles of the coefficients give the \textit{singularities} of the equation. These singularities are $s=0$ (at the axis) and $s=1$ ($H=0$, at the equator), and as will be seen in other case studies, these are the same for all cases presented.
Note $H = (1 - s)^{1/2} (1 + s)^{1/2}$. Therefore, a denominator in the form of $s^a H^b$ produces an $a-$th order pole $s=0$, and an $\frac{b}{2}$-th order pole $s=1$. An integer $b$ that is odd produces an \textit{essential singularity} at $s=1$.

Recalling the properties of ODEs, an $n-$th order ODE
\[
    \frac{d^n y}{dx^n} + \sum_{k=0}^{n-1} a_k(x) \frac{d^k y}{dx^k} = 0
\]
admits regular solutions in the vicinity of $x=x_0$ so long as the following quantities are analytic:
\[
    (x - x_0)^k a_{n-k}(x),\qquad k = 0, 1, \cdots n-1.
\]
In other words, $a_{n-k}(x)$ is allowed to have a pole up to the $k-$th order. For the second-order ODE above, $a_1$ has simple poles at $s=0$ and $s=1$, and $a_0$ has a second-order pole at $s=0$, a simple pole at $s=1$. Therefore, all singular points of the coefficients are merely \textit{regular singular points} of the equation, or \textit{apparent singularities}, and the existence of regular solution is guaranteed.

\subsection{Analytical solution}

The hydrodynamic equation (\ref{eqn:ode-hydro}) has known analytical solutions. The eigenvalues are given by
\begin{equation}
    \widetilde{\omega}_n^m = \omega_n^m = \frac{-m}{(n + 1)(2n + 2m + 3) + \frac{m}{2} + \frac{m^2}{4}}, \quad 
    \psi^{m}_n(s) = s^m H^3 P_{n}^{\left(\frac{3}{2}, m\right)}(2s^2 - 1),\quad n \in \mathbb{Z}^*.
\end{equation}
This is a rare case where the eigenvalue and eigenfunction can be obtained in closed form.
These eigenmodes are the inertial modes. Forming a complete (and orthonormal) set in the appropriate Hilbert space, they provide a basis for the streamfunction in the columnar ansatz. The current implementation of the PG model uses these as the radial spectral basis for $\psi^m(s)$.

The analytical solution also indicates that the linear operator
\[
    \frac{H^3}{s} \left[\frac{d}{ds}\left(\frac{s}{H} \frac{d}{ds}\right) - \frac{m^2}{sH}\right] = (1 - s^2) \frac{d^2}{ds^2} + \frac{1}{s}\frac{d}{ds} - m^2 \frac{1 - s^2}{s^2}
\]
has eigenvalues $\lambda_n' = - 2(n+1)(2n+2m+3) - m$, with corresponding eigenfunctions $\psi_n^m$ as stated above.

\subsection{System spectrum}

The problem is solved numerically using the spectral PG code \texttt{PlesioGeostroPy} for several $m$.
\begin{figure}[htbp]
    \centering
    \includegraphics[width=.8\linewidth]{../../out/eigen/Hydrodynamic/Analytical_error.pdf}
    \caption{Eigenperiods for $m=3$ modes solved using transformed variables and reduced system, with analytic solutions. Lower panel shows the relative error compared to analytical solutions.}
    \label{fig:eigenperiod-hydro-m3}
\end{figure}

The eigenvalues for $m=3$ eigenmodes are presented in Fig.(\ref{fig:eigenperiod-hydro-m3}). The quadratures are computed in double precision, and the matrices are inserted into a double precision eigensolver.
Both the results of the full system (Transformed variables) and the reduced system are presented. These are both solved using a truncation level of $50$ for the streamfunction $\psi$.

All hydrodynamic eigenmodes, or inertial modes in the PG model, are eastwards modes (Fig.\ref{fig:complex-spectrum-hydro-m3}). As expected from physical arguments, the real parts of the numerically solved eigenvalues are close to machine precision from zero (Fig.\ref{fig:complex-spectrum-hydro-m3}).
The very small discrepancy from the analytical values (unanimously lower than $10^{-14}$, lower panel of Fig.\ref{fig:eigenperiod-hydro-m3}) indicates that all 51 eigenvalues are solved satisfactorily close to machine precision.
This is unsurprising since the spectral basis used for the streamfunction is nothing but the analytical eigenmodes, yielding perfect convergence. For this very simple problem, there is virtually no difference between the eigenvalues solved using reduced system or full system, as both are virtually accurate down to machine precision.
\begin{figure}[htbp]
    \centering
    \includegraphics[width=.8\linewidth]{../../out/eigen/Hydrodynamic/Spectrum_m3_Le1e-4.pdf}
    \caption{Complex spectrum of the $m=3$ eigenvalues.}
    \label{fig:complex-spectrum-hydro-m3}
\end{figure}

Taking a step back, the eigenperiods of the fundamental ($n=0$), 2nd-, 5th- and 9th-order modes are shown in Fig.(\ref{fig:period-m-hydro}) as a function of azimuthal wavenumber $m$.
This is basically a reproduction of Fig.(4.1) in \textcite{holdenried-chernoff_long_2021} and Fig.(1) in \textcite{jackson_plesio-geostrophy_2020}, except the current plot shows numerically solved eigenperiods, while the plots in the cited ones are probably just analytical solutions.
\begin{figure}[htbp]
    \centering
    \includegraphics[width=.8\linewidth]{../../out/eigen/Hydrodynamic/Period_wavenumber_plot_m20_n10.pdf}
    \caption{Periods as a function of azimuthal wavenumber for different order modes.}
    \label{fig:period-m-hydro}
\end{figure}

There is a discrepancy between the \textcite{jackson_plesio-geostrophy_2020} periods and the \textcite{holdenried-chernoff_long_2021} periods. The former one is mostly likely missing a $2\pi$ factor, while the latter one is consistent with the numerical results presented here.The shortest eigenperiod is observed in the fundamental mode for $m=3$, which has a period of circa $26.7$ days. The eigenperiods of all azimuthal wavenumbers increase with higher orders. 


\subsection{Selected eigenmodes}

The hydrodynamic eigenmodes are pure and simple.
As mentioned, they take the analytical form
\[
    \psi^{mn} = s^{|m|}H^3 P_n^{\left(\frac{3}{2}, |m|\right)}(2s^2 - 1)\, e^{im\phi}.
\]
Several eigenmodes are visualized and their spectra shown below.
For simplicity, only the results solved using the reduced system are used. However, we have already seen from the eigenvalue comparisons that the full system yields virtually the same solution, at least in this simple eigenvalue problem.

\begin{figure}[htbp]
    \centering
    \begin{subfigure}[b]{\linewidth}
        \includegraphics[width=\linewidth]{../../out/eigen/Hydrodynamic/Reduced/mode_equatorial_m3n0.png}
    \end{subfigure}
    \begin{subfigure}[b]{\linewidth}
        \includegraphics[width=\linewidth]{../../out/eigen/Hydrodynamic/Reduced/mode_meridional_m3n0.png}
    \end{subfigure}
    \caption{Fundamental ($n=0$) hydrodynamic eigenmode for $m=3$. The upper panel and the lower panel show the equatorial ($z=0$) and meridional ($\phi=\pi/4$) slices, respectively.}
    \label{fig:eigenmode-hydro-m3n0}
\end{figure}

\begin{figure}[htbp]
    \centering
    \includegraphics[width=\linewidth]{../../out/eigen/Hydrodynamic/Reduced/spectrum_m3n0.pdf}
    \caption{Fundamental $m=3$ eigenmode streamfunction spectrum at different truncation levels.}
    \label{fig:modespec-hydro-m3n0}
\end{figure}

The fundamental and the 10-th eigenmode for azimuthal wavenumber $m=3$ are visualized in Figs.(\ref{fig:eigenmode-hydro-m3n0}) and (\ref{fig:eigenmode-hydro-m3n10}), respectively. The visualized amplitude is normalized such that the streamfunction $\psi$ has amplitude unity in the equatorial plane.
Even in the fundamental mode, we see that the azimuthal velocity near the equator is much stronger than any components anywhere else.
This is even more exaggerated for higher order modes. As a result, the vorticity is also increasingly concentrated near the equator for higher order modes.
Readers who find this result surprising should read section 1.4 of the Ingredients document, where the analytical bases for velocity components as well as vorticity are derived. The analytical bases show that higher order basis function for $\psi$ translates to strong azimuthal velocity and concentrated vorticity at the equator.

\begin{figure}[htbp]
    \centering
    \begin{subfigure}[b]{\linewidth}
        \includegraphics[width=\linewidth]{../../out/eigen/Hydrodynamic/Reduced/mode_equatorial_m3n10.png}
    \end{subfigure}
    \begin{subfigure}[b]{\linewidth}
        \includegraphics[width=\linewidth]{../../out/eigen/Hydrodynamic/Reduced/mode_meridional_m3n10.png}
    \end{subfigure}
    \caption{10-th hydrodynamic eigenmode for $m=3$. The upper panel and the lower panel show the equatorial ($z=0$) and meridional ($\phi=\pi/4$) slices, respectively.}
    \label{fig:eigenmode-hydro-m3n10}
\end{figure}

\begin{figure}[htbp]
    \centering
    \includegraphics[width=\linewidth]{../../out/eigen/Hydrodynamic/Reduced/spectrum_m3n10.pdf}
    \caption{10-th $m=3$ eigenmode streamfunction spectrum.}
    \label{fig:modespec-hydro-m3n10}
\end{figure}

As alluded to, the eigenmodes coincide with the spectral basis (or rather the other way around: the spectral basis comes from the hydrodynamic eigenmodes). This means that the spectrum will contain only spikes at selected basis, as is the case in Figs.(\ref{fig:modespec-hydro-m3n0}) and (\ref{fig:modespec-hydro-m3n10}).
The fundamental and 10-th eigenmode only has nontrivial coefficient corresponding to the 0-th and the 10-th spectral basis, while all other coefficients are within 5 times machine precision from zero.
The result is that the convergence is perfect, meaning as soon as the necessary basis is included within the truncation level, the problem is exactly solvable.

\clearpage


\section{Ideal eigenmodes under Malkus background field}

The background field is given by
\[
    \mathbf{B}^0 = s \hat{\bm{\phi}}
\]
translating to the following background PG quantities:
\begin{equation}
\begin{aligned}
    \overline{M_{\phi\phi}}^0 &= 2 s^{2} H \\ 
    \widetilde{zM_{\phi\phi}}^0 &= s^{2} H^{2} \\ 
    B_{\phi}^{0e} &= s \\ 
    B_\phi^{0+} &= s \\
    B_\phi^{0-} &= s \\
    \Psi^{0} &= \overline{M_{ss}}^0 = \overline{M_{s\phi}}^0 
    = \widetilde{M_{sz}}^0 = \widetilde{M_{\phi z}}^0 = \widetilde{zM_{ss}}^0 = \widetilde{zM_{s\phi}}^0 = 0 \\
    B_{s}^{0e} &= B_{z}^{0e} = B_{s, z}^{0e} = B_{\phi, z}^{0e} = B_s^{0+} = B_z^{0+} = B_s^{0-} = B_z^{0-} = 0
\end{aligned}
\end{equation}

\subsection{Linearized equations}

In the Fourier domain (ansatz $\psi = \psi(s) e^{i\widetilde{\omega}t + im\phi} = \psi(s)e^{\lambda t + im\phi}$), the set of linearized equations under the Malkus background field reads
\begin{equation}\label{eqn:eigen-pg-malkus}
\begin{aligned}
    i\widetilde{\omega} \left[\frac{d}{d s}\left(\frac{s}{H}\frac{d}{d s}\right) - m^2 \left(\frac{1}{sH} + \frac{s}{2H^3}\right)\right] \psi &= \frac{2 i m s}{\mathrm{Le} H^{3}} \psi \\
    &\mkern-260mu + \frac{i m}{2 H} \left( \frac{d \overline{m_{ss}}}{d s} - \frac{d \overline{m_{\phi\phi}}}{d s}\right) + \frac{i m}{2 sH} \left(\overline{m_{ss}} - \overline{m_{\phi\phi}}\right) - \frac{s}{2 H} \frac{d^{2} \overline{m_{s\phi}}}{d s^{2}} - \frac{3}{2 H} \frac{d\overline{m_{s\phi}}}{d s} - \frac{m^{2}}{2 H s} \overline{m_{s\phi}} \\
    &\mkern-260mu - \frac{i m s}{2 H^{2}} \frac{d \widetilde{m_{sz}}}{d s} + \frac{m^{2}}{2 H^{2}} \widetilde{m_{\phi z}} - \frac{i m}{2 H^{2}} \widetilde{m_{sz}} - \frac{i m s^{2}}{H^{2}} b_{\phi}^e - \frac{2 s^{2}}{H^{2}}b_{s}^e \\
    &\mkern-260mu - \frac{s^{3}}{2 H^{2}} \left(\frac{db^{+}_{s}}{d s} + \frac{d b^{-}_{s}}{ds}\right) + \left(- \frac{3 s^{2}}{2 H^{2}} - \frac{s^{4}}{2 H^{4}}\right) \left( b^{+}_{s} + b^{-}_{s} \right) - \frac{s^{2}}{2 H} \left(\frac{d b^{+}_{z}}{d s} - \frac{d b^{-}_{z}}{d s}\right) - \frac{s}{H} \left(b^{+}_{z} - b^{-}_{z}\right)\\
    i \widetilde{\omega} \overline{m_{\phi\phi}} &= - 4 i m s \frac{d \psi}{d s} \\
    i \widetilde{\omega} \overline{m_{s\phi}} &= - 2 m^{2} \psi \\
    i \widetilde{\omega} \widetilde{m_{\phi z}} &= \frac{m^{2} s}{H} \psi \\
    i \widetilde{\omega} \widetilde{zm_{\phi\phi}} &= - 2 i m s H \frac{d \psi}{d s} \\
    i \widetilde{\omega} \widetilde{zm_{s\phi}} &= - m^{2} H \psi\\
    i \widetilde{\omega} b_{s}^e &= - \frac{m^{2}}{s H} \psi \\
    i \widetilde{\omega} b_{\phi}^e &= - \frac{i m}{H} \frac{d \psi}{d s} \\
    \overline{m_{ss}} = \widetilde{zm_{ss}} = \widetilde{m_{sz}} &= b_{z}^e = b_{s, z}^e = b_{\phi, z}^e = 0
\end{aligned}
\end{equation}
with induction equations at the boundary given by
\begin{equation}\label{eqn:eigen-bound-malkus}
    \begin{aligned}
        i \widetilde{\omega} b^{+}_{s} &= i \widetilde{\omega} b^{-}_{s} = - \frac{m^{2}}{s H} \psi\\
        i \widetilde{\omega} b^{+}_{\phi} &= i \widetilde{\omega} b^{-}_{\phi} = - \frac{i m}{H} \frac{d \psi}{d s}\\
        i \widetilde{\omega} b^{+}_{z} &= -i \widetilde{\omega} b^{-}_{z} = - \frac{m^{2}}{H^{2}} \psi.
    \end{aligned}
\end{equation}
The induction equations in (\ref{eqn:eigen-pg-malkus}) and (\ref{eqn:eigen-bound-malkus}) have been validated against and are indeed exactly the same as those reported in \textcite{holdenried-chernoff_long_2021}.
There is no reference for the streamfunction equation, but its validity can be partially checked when the equation is further reduced into lower dimensions, as shown in the next subsection.

As a final side remark, note that the boundary magnetic terms in the momentum equation (\ref{eqn:eigen-pg-malkus}) either consist of sums of or differences between the upper boundary and the lower boundary terms, as a result of the symmetry of the background field with respect to the equatorial plane.
The same parity leads to the boundary terms as dictated by (\ref{eqn:eigen-bound-malkus}) to be either odd or even functions.
The overall outcome is that the boundary terms cancel each other out, and have no effects in the system whatsoever.
Although not explicitly pointed out in \textcite{holdenried-chernoff_long_2021}, the original implementation in \texttt{Mathematica} ignores the contribution of the boundary terms, but nevertheless yields the correct output. This is most likely due to the fact that the boundary terms play no role anyway in the Malkus model.

\subsection{Standard ODE form}

The Malkus field, despite its complicated momentum equation, has a particularly simple reduced form.
The 2-order form of the dynamical system takes the following form,
\begin{equation}
\begin{aligned}
    i\widetilde{\omega} \left(1 - \frac{m^2}{\widetilde{\omega}^2}\right) \left[\frac{d}{d s}\left(\frac{s}{H}\frac{d}{d s}\right) - m^2 \left(\frac{1}{sH} + \frac{s}{2H^3}\right)\right] \psi &= 2i \left(\frac{m}{\mathrm{Le}} - \frac{m^2}{\widetilde{\omega}}\right)\frac{s}{H^3} \psi \\ 
    \left[\frac{d}{d s}\left(\frac{s}{H}\frac{d}{d s}\right) - m^2 \left(\frac{1}{sH} + \frac{s}{2H^3}\right)\right] \psi &= 2 \frac{\frac{1}{\mathrm{Le}} m\widetilde{\omega} - m^2}{\widetilde{\omega}^2 - m^2}\frac{s}{H^3} \psi
\end{aligned}
\end{equation}
which only differs from the hydrodynamic case (\ref{eqn:ode-hydro}) by a factor. 
\[
    \frac{d^{2}}{d s^{2}} \psi + \frac{1}{sH^{2}} \frac{d\psi}{d s} + \frac{m}{m^2 - \widetilde{\omega}^2} \frac{1}{2s^2H^2} \left(\frac{4\widetilde{\omega}s^2}{\mathrm{Le}} + m(m^2 - \widetilde{\omega}^2)(s^2 -2) - 4ms^2\right) \psi = 0
\]
At the current stage, this form really doesn't seem to yield more information than the fact that both $s=0$ and $s=1$ are regular singularities of the equation, and regular solutions should exist.

\subsection{Analytical solution}
As the ODE in $\psi$ only differs from eq.(\ref{eqn:ode-hydro}) by a factor, the Malkus background field also inherits the analytical solution from the hydrodynamic case.
The analytical eigenvalues are calculated from
\[
    \frac{\frac{1}{\mathrm{Le}} m\widetilde{\omega} - m^2}{\widetilde{\omega}^2 - m^2} = \frac{m}{\omega_{\mathrm{hydro}}^{mn}}
\]

\clearpage



\section{Eigenmodes in the toroidal quadrupolar background field}

\clearpage

\section{Ideal eigenmodes under poloidal dipolar background field}

The background field is given by
\[
    \mathbf{B}^0 = -6sz \hat{\mathbf{s}} + (12s^2 + 6z^2 - 10) \hat{\mathbf{z}}
\]
which translates to the following background PG quantities
\begin{equation}
\begin{aligned}
    \overline{M_{ss}}^0 &= 24 H^{3} s^{2} \\ 
    \widetilde{M_{sz}}^0 &= - 18 H^{4} s + 2 H^{2} \left(- 36 s^{3} + 30 s\right) \\
    \widetilde{zM_{ss}}^0 &= 18 H^{4} s^{2} \\
    B_{z}^{0e} &= 12 s^{2} - 10 \\
    B_{s, z}^{0e} &= - 6 s \\
    B_s^{0+} &= - 6 H s \\
    B_z^{0+} &= 6 s^{2} - 4 \\
    B_s^{0-} &= 6 H s \\
    B_z^{0-} &= 6 s^{2} - 4 \\
    \Psi^{0} &= \overline{M_{\phi\phi}}^0 = \overline{M_{s\phi}}^0 = \widetilde{M_{\phi z}}^0 = \widetilde{zM_{\phi\phi}}^0 = \widetilde{zM_{s\phi}}^0 = 0 \\ 
    B_{s}^{0e} &= B_{\phi}^{0e} = B_{\phi, z}^{0e} = B_\phi^{0+} = B_\phi^{0-} = 0.
\end{aligned}
\end{equation}

\subsection{Linearized equations}

As always, the most complicated part is the streamfunction equation. Under the poloidal dipolar background field, it takes the form,
\begin{equation}\label{eqn:eigen-pgpsi-poloidal-dipole}
\begin{aligned}
    i\widetilde{\omega} \left[\frac{d}{d s}\left(\frac{s}{H}\frac{d}{d s}\right) - m^2 \left(\frac{1}{sH} + \frac{s}{2H^3}\right)\right] \psi &= \frac{2 i m s}{\mathrm{Le} H^{3}} \psi \\
    &\mkern-240mu + \frac{i m}{2 H} \frac{d\overline{m_{ss}}}{d s} + \frac{i m}{2 H s} \overline{m_{ss}} - \frac{i m}{2 H} \frac{d \overline{m_{\phi\phi}}}{d s} - \frac{i m}{2 H s} \overline{m_{\phi\phi}} - \frac{s}{2 H} \frac{d^{2}\overline{m_{s\phi}}}{d s^{2}} - \frac{3}{2 H} \frac{d\overline{m_{s\phi}}}{d s} - \frac{m^{2}}{2 sH} \overline{m_{s\phi}} \\
    &\mkern-240mu - \frac{i m s}{2 H^{2}} \frac{d\widetilde{m_{sz}}}{d s} - \frac{i m}{2 H^{2}}\widetilde{m_{sz}} + \frac{m^{2}}{2 H^{2}} \widetilde{m_{\phi z}} + 4 i ms\frac{6 s^{2} - 5}{H^{2}} b_{z}^e - 2s^2\frac{6 s^{2} - 5}{H^{2}} b_{\phi, z}^e \\
    &\mkern-240mu + \frac{2 s}{H} \left(\frac{db^{+}_{\phi}}{d s} - \frac{db^{-}_{\phi}}{d s}\right) + \frac{2}{H} \left(b^{+}_{\phi} - b^{-}_{\phi}\right) + \frac{i m s}{H^{2}} \left(b^{+}_{z} + b^{-}_{z}\right) + \frac{i m \left(s^2 - 2\right)}{H^{3}} \left(b^{+}_{s} - b^{-}_{s}\right).
\end{aligned}
\end{equation}
The magnetic quantities are described by the induction equations in the PG model, which are given by
\begin{equation}\label{eqn:eigen-pgmag-poloidal-dipole}
\begin{aligned}
    i \widetilde{\omega} \overline{m_{ss}} &= i48m \left[ sH^2 \frac{d\psi}{d s} + \left(4 s^{2} - 2\right) \psi\right] \\ 
    % i \widetilde{\omega} \overline{m_{s\phi}} &= - 24 H^{2} s^{2} \frac{d^{2}}{d s^{2}} \psi + \left(24 H^{2} s - 24 s^{3}\right) \frac{d}{d s} \psi \\
    i \widetilde{\omega} \overline{m_{s\phi}} &= 24 \left[-s^2 H^{2} \frac{d^{2} \psi}{d s^{2}} + s\left(1-2s^2\right) \frac{d\psi}{d s}\right] \\
    % i \widetilde{\omega} \widetilde{m_{sz}} &= \left(- \frac{324 i m s^{3}}{H} + \frac{342 i m s}{H} - \frac{84 i m}{H s}\right) \psi + \left(\frac{72 i m s^{4}}{H} - \frac{114 i m s^{2}}{H} + \frac{42 i m}{H}\right) \frac{d}{d s} \psi \\
    i \widetilde{\omega} \widetilde{m_{sz}} &= i6m \left[\frac{12s^4 - 19s^2 + 7}{H} \frac{d\psi}{d s} - \frac{54s^4 - 57s^2 + 14}{sH} \psi\right] \\
    % i \widetilde{\omega} \widetilde{m_{\phi z}} &= \left(54 H s^{3} - 42 H s\right) \frac{d^{2}}{d s^{2}} \psi + \left(- 54 H s^{2} + 42 H + \frac{54 s^{4}}{H} - \frac{42 s^{2}}{H}\right) \frac{d}{d s} \psi \\
    i \widetilde{\omega} \widetilde{m_{\phi z}} &= 6 \left[sH \left(9 s^{2} - 7\right) \frac{d^{2}\psi}{d s^{2}} + \frac{1}{H}\left(2s^2 - 1\right)\left(9s^2 - 7\right) \frac{d\psi}{d s}\right] \\
    % i \widetilde{\omega} \widetilde{zm_{ss}} &= 36 i H^{3} m s \frac{d}{d s} \psi + \left(- 36 i H^{3} m + 108 i H m s^{2} - 36 i H m\right) \psi \\
    i \widetilde{\omega} \widetilde{zm_{ss}} &= i36 m \left[s H^{3} \frac{d\psi}{d s} + 2H \left(2 s^{2} - 1 \right) \psi\right] \\
    % i \widetilde{\omega} \widetilde{zm_{s\phi}} &= - 18 H^{3} s^{2} \frac{d^{2}}{d s^{2}} \psi + \left(18 H^{3} s - 18 H s^{3}\right) \frac{d}{d s} \psi \\
    i \widetilde{\omega} \widetilde{zm_{s\phi}} &= 18 \left[- s^2 H^{3} \frac{d^{2}\psi}{d s^{2}} + sH \left(1 - 2s^2\right) \frac{d \psi}{d s} \right] \\
    % i \widetilde{\omega} b_{z}^e &= \left(\frac{12 i m s^{2}}{H^{3}} - \frac{14 i m}{H^{3}}\right) \psi \\
    i \widetilde{\omega} b_{z}^e &= i2m \frac{6s^2 - 7}{H^3} \psi \\
    % i \widetilde{\omega} b_{s, z}^e &= \left(\frac{12 i m}{H s} - \frac{12 i m s}{H^{3}}\right) \psi - \frac{6 i m \frac{d}{d s} \psi}{H} \\
    i \widetilde{\omega} b_{s, z}^e &= i6m \left[- \frac{1}{H}\frac{d\psi}{d s} + \frac{2(1 - 2s^2)}{sH^3} \psi\right] \\
    % i \widetilde{\omega} b_{\phi, z}^e &= \left(- \frac{6}{H} + \frac{6 s^{2}}{H^{3}}\right) \frac{d}{d s} \psi + \frac{6 s \frac{d^{2}}{d s^{2}} \psi}{H} \\
    i \widetilde{\omega} b_{\phi, z}^e &= 6\left[\frac{s}{H}\frac{d^{2}\psi}{d s^{2}} + \frac{2s^2 - 1}{H^3} \frac{d\psi}{d s} \right] \\
    \overline{m_{\phi\phi}} &= \widetilde{zm_{\phi\phi}} = b_{s}^e = b_{\phi}^e = 0
\end{aligned}
\end{equation}
All the PG induction equations are validated against and are exactly the same as in \textcite{holdenried-chernoff_long_2021}.
In contrast, the boundary induction equations are given as follows
\begin{equation}\label{eqn:eigen-bound-poloidal-dipole}
\begin{aligned}
    i \widetilde{\omega} b^{+}_{s} &= - i \widetilde{\omega} b^{-}_{s} = i 6m \left[-\frac{d\psi}{d s} + \frac{2(1 - 2s^2)}{sH^2} \psi\right], \\
    % i \widetilde{\omega} b^{+}_{\phi} &= -i \widetilde{\omega} b^{-}_{\phi} = 6 s \frac{d^{2}}{d s^{2}} \psi + \left(-6 + \frac{6 s^{2}}{H^{2}}\right) \frac{d}{d s} \psi \\
    i \widetilde{\omega} b^{+}_{\phi} &= -i \widetilde{\omega} b^{-}_{\phi} = 6 \left[s \frac{d^{2}\psi}{d s^{2}} + \frac{2s^2 - 1}{H^2} \frac{d\psi}{d s} \right], \\ 
    i \widetilde{\omega} b^{+}_{z} &= i \widetilde{\omega} b^{-}_{z} = i2m \left[\frac{3 s}{H} \frac{d\psi}{d s} + \frac{4(3s^2 - 1)}{H^3} \psi \right].
\end{aligned}
\end{equation}
While the meridional components $b_s^{\pm}$ and $b_z^{\pm}$ share the same expression as in \textcite{holdenried-chernoff_long_2021}, the azimuthal components $b_\phi^{\pm}$ are not mentioned in the previous work.
It is unclear what argument was made to ignore this boundary term, but the calculation here seems to suggest that this component is however not trivial.
Is this used or not in the previously obtained results? This is a question that only the original author can answer.

We see that contrary to the boundary fields for Malkus field, the boundary fields for the poloidal dipolar background field has another parity.
The equatorial components $b_s$ and $b_\phi$ are opposite at the upper and the lower boundary, while the vertical component $b_z$ matches.
This means that the boundary terms in (\ref{eqn:eigen-pgpsi-poloidal-dipole}) do not cancel out like they do in (\ref{eqn:eigen-pg-malkus}), and thus boundary terms do matter.


\subsection{Standard ODE form}

To check the regularity of the system, we form the standard ODE of $\psi$ as a function of $s$ by merging the PG equations and boundary induction equations into one single ODE (\ref{eqn:ode-poloidal-dipole}).
The standard form is a 4-th order system in $\psi$. The singularities of the equations are $s=0$, i.e. at the axis, and $H=0$ ($s=1$), i.e. at the equator. The singularities of the coefficients are analyzed below.
\begin{equation}\label{eqn:ode-poloidal-dipole}
\begin{aligned}
    &\frac{d^{4}\psi}{d s^{4}} + \frac{8 - 9 s^{2}}{sH^{2}} \frac{d^{3}\psi}{d s^{3}} \\
    +& \left(\frac{\omega^{2}}{12 H^{2} s^{2}} + \frac{- m^{2} \left(s^{4} - 5 s^{2} + 4\right) - 16 s^{2} + 20}{4 s^{2} H^{4}}\right) \frac{d^{2}\psi}{d s^{2}} \\
    +& \left(\frac{\omega^{2}}{12 s^{3} H^{4}} + \frac{- m^{2} \left(s^{6} - 2 s^{4} - 2 s^{2} + 3\right) + 18 s^{6} - 51 s^{4} + 41 s^{2} - 5}{s^3 H^{6}}\right) \frac{d\psi}{d s} \\
    +& \left(\frac{\omega^{2} m^{2} \left(s^2 - 2\right)}{24 s^{4} H^{4}} - \frac{1}{\mathrm{Le}} \frac{\omega m}{6 s^{2} H^{4}} - \frac{m^{2} \cdot \left(30 s^{6} - 127 s^{4} + 156 s^{2} - 32\right)}{4 s^{4} H^{6}}\right) \psi = 0
\end{aligned}
\end{equation}
% \begin{itemize}
%     \item The 3-rd order term has a simple pole at $s=0$ and a simple pole at $s=1$.
%     \item The 2-nd order term has a 2-nd order pole at $s=0$ and a 2-nd order pole at $s=1$.
%     \item The 1-st order term has a 3-rd order pole at $s=0$ and a 3-rd order pole at $s=1$.
%     \item The 0-th order term has a 4-th order pole at $s=0$ and a 3-rd order pole at $s=1$.
% \end{itemize}

The 3-rd order term has a simple pole at $s=0$ and a simple pole at $s=1$;

The 2-nd order term has a 2-nd order pole at $s=0$ and a 2-nd order pole at $s=1$;

The 1-st order term has a 3-rd order pole at $s=0$ and a 3-rd order pole at $s=1$;

The 0-th order term has a 4-th order pole at $s=0$ and a 3-rd order pole at $s=1$.

The overall conclusion is that the singularities of the equation are both regular singularities, so regular solutions should exist.
However, singularity aside, these coefficients differ from the ones as provided in \textcite{holdenried-chernoff_long_2021}. Some coefficients are the same, but not all. Recalling that in the previous work, the $b_\phi^{\pm}$ may be ignored, I tried to remove these equations before assembling the 2-nd order ODE. The result is as follows
\begin{equation}
\begin{aligned}
    &\frac{d^{4}\psi}{d s^{4}} + \frac{6 - 9 s^{2}}{sH^{2}} \frac{d^{3}\psi}{d s^{3}} \\
    +& \left(\frac{\omega^{2}}{12 s^{2} H^{2}} + \frac{m^{2} \left(- s^{4} + 5 s^{2} - 4\right) - 16 s^{2} + 12}{4 s^{2} H^{4}}\right) \frac{d^{2}\psi}{d s^{2}} \\
    +& \left(\frac{\omega^{2}}{12 s^{3} H^{4}} + \frac{- m^{2} \left(s^{6} - 2 s^{4} - 2 s^{2} + 3\right) + 18 s^{6} - 47 s^{4} + 31 s^{2} - 3}{s^{3} H^{6}}\right) \frac{d\psi}{d s} \\
    +& \left(\frac{\omega^{2} m^{2} \left(s^2 - 2\right)}{24 s^{4} H^{4}} - \frac{1}{\mathrm{Le}} \frac{\omega m}{6 s^{2} H^{4}} - \frac{m^{2} \cdot \left(30 s^{6} - 127 s^{4} + 156 s^{2} - 32\right)}{4 s^{4} H^{6}}\right) \psi = 0
\end{aligned}
\end{equation}
Now the first three coefficients all match with \textcite{holdenried-chernoff_long_2021}. However, the coefficients of the lowest two order terms still do not match. The only difference, as it seems, lies in the terms containing $m^2$ in these coefficients.
It does not seem to be possible to tell the right from wrong here, so a thorough symbolic re-analysis seems necessary.

One notable feature that might aid debugging is that according to the equations presented here, there aren't even terms that can give $s^7$ on the numerator. However, both discrepant coefficients contain $s^7m^2$ terms in \textcite{holdenried-chernoff_long_2021}. "Dimension" analysis of the streamfunction equation (\ref{eqn:eigen-pgpsi-poloidal-dipole}) seems to indicate that such a high degree is not possible. If something is wrong with the derivation, the mistake may have already occurred at eq.(\ref{eqn:eigen-pgpsi-poloidal-dipole}). Understanding how this $s^7$ term comes about might be a key to finding out potential errors.


\subsection{System spectrum}

\begin{figure}[htbp]
    \centering
    \includegraphics[width=.9\linewidth]{../../out/eigen/Poloidal_Dipole/Reduced/Spectrum_m3_Le-1.414e-4.pdf}
    \caption{Spectrum for $m=3$ modes.}
\end{figure}

\subsection{Selected eigenmodes}

\clearpage

