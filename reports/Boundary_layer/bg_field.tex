\documentclass[a4paper, 11pt]{article}
\usepackage{doc_default}
\usepackage{newtxtext, newtxmath}
\usepackage[
    backend=biber,
    style=authoryear-comp,
    dashed=false,
    % compact=true,
    sorting=nyt
]{biblatex}
% \addbibresource{../references.bib}

\usepackage{hyperref}
\hypersetup{
    colorlinks=true,
    linkcolor=blue,
    citecolor=blue,
    urlcolor=blue
}

% \DeclareMathOperator{\arcsec}{arcsec}
% \DeclareMathOperator{\arccot}{arccot}
% \DeclareMathOperator{\arccsc}{arccsc}
\DeclareMathOperator{\sgn}{sgn}

\newcommand{\todoitem}[1]{\textcolor{purple}{[#1]}}

\title{Fabrication of a magnetic background field radial at the boundary}
\author{Jingtao Min}
\date{Sept 26, 2024, last update \today}

\begin{document}

\maketitle

\begin{abstract}
    \noindent This document aims to construct magnetic background fields in a sphere that are radial at the boundary (i.e., $\hat{\mathbf{r}}\times \mathbf{B}|_{r = a} = \mathbf{0}$). Such fields can be useful for testing boundary layer parameterisations.
\end{abstract}

\section{Boundary-normal magnetic field}

For a magnetic field $\mathbf{B}$ defined in a sphere $r \leq a$, we want to see what conditions its parameterisation must satisfy in order that it is radial at the boundary. In other words, its tangential components must vanish at the boundary, i.e. 
\begin{equation}
    \hat{\mathbf{r}}\times \mathbf{B}|_{r=a} = \mathbf{0}.
\end{equation}
Since magnetic field is solenoidal ($\nabla\cdot \mathbf{B} = 0$) by nature (or according to Maxwell equations), we can enforce the divergence-free property easily by utilising the Toroidal-Poloidal representation (Mie representation). This takes the form
\begin{equation}
    \mathbf{B} = \nabla\times T \mathbf{r} + \nabla\times \nabla\times S \mathbf{r} = - \hat{\mathbf{r}} \nabla_s^2 \frac{S}{r} + \nabla_s \left(\frac{1}{r}\frac{\partial}{\partial r}(rS)\right) - \hat{\mathbf{r}}\times \nabla_s T
\end{equation}
where $T(\mathbf{r}) = T(r,\theta,\phi)$ and $S(\mathbf{r}) = S(r,\theta,\phi)$ are the toroidal and poloidal scalars, respectively. $\nabla_s$ is the surface gradient operator on the surface of a sphere, and can be defined from normal gradient via $\nabla_s = r (\nabla - \hat{\mathbf{r}} \hat{\mathbf{r}}\cdot \nabla) = r (\nabla - \hat{\mathbf{r}} \partial_r)$, and takes the explicit form $\nabla_s = \hat{\bm{\theta}} \partial_\theta + \hat{\bm{\phi}} (\sin\theta)^{-1} \partial_\phi$ in spherical coordinates; 
$\nabla_s^2$ is the surface Laplacian, and can be defined from normal gradient via $\nabla_s^2 = r^2 (\nabla^2 - r^{-2}\partial_r (r^2 \partial_r)) = r^2 \nabla^2 - \partial_r (r^2 \partial_r)$, and takes the explicit form $\nabla_s^2 = (\sin\theta)^{-1}\partial_\theta (\sin\theta \partial_\theta) + (\sin\theta)^{-2} \partial_\phi^2$ in spherical coordinates.
We expand the two scalars in spherical harmonics,
\begin{equation}
    S = \sum_{l,m} S_{lm}(r) Y_{lm}(\theta, \phi),\qquad 
    T = \sum_{l,m} T_{lm}(r) Y_{lm}(\theta, \phi).
\end{equation}
Using the commutation relation of $\nabla_s^2$ and $\nabla_s$, and the fact that $\nabla_s^2 Y_l^m = -l(l+1)$, the magnetic field can then be rewritten as
\begin{equation}
    \mathbf{B} = \sum_{l,m} \left[\frac{l(l+1)}{r} S_{lm}(r) Y_l^m(\theta, \phi) \hat{\mathbf{r}} + 
    \frac{1}{r}\frac{d(rS_{lm}(r))}{dr} \nabla_s Y_l^m(\theta, \phi) - T_{lm}(r) \hat{\mathbf{r}}\times \nabla_s Y_l^m(\theta, \phi)\right]
\end{equation}
Inserting this into the boundary condition $\hat{\mathbf{r}}\times \mathbf{B}|_{r=a} = \mathbf{0}$, we have
\begin{equation}
\begin{aligned}
    \hat{\mathbf{r}}\times \mathbf{B}|_{r=a} &= \sum_{l,m} \left[\frac{1}{r}\frac{d(rS_{lm}(r))}{dr} \hat{\mathbf{r}}\times \nabla_s Y_l^m(\theta, \phi) + T_{lm}(r) \nabla_s Y_l^m(\theta, \phi) \right]_{r=a} \\ 
    &= \sum_{l,m} \left[\frac{1}{a}\frac{d(rS_{lm}(r))}{dr}\Big|_{a} \mathbf{C}_l^m(\theta, \phi) + T_{lm}(a) \mathbf{B}_l^m(\theta, \phi) \right] = \mathbf{0}
\end{aligned}
\end{equation}
where $\mathbf{B}_l^m = \nabla_s Y_l^m$ and $\mathbf{C}_l^m = \hat{\mathbf{r}}\times \nabla_s Y_l^m$ are the vector spherical harmonics. These vector bases are orthogonal for different $l$ and $m$, and also mutually orthogonal. Therefore, for the series to identically vanish, the coefficients for $\mathbf{B}_l^m$ and $\mathbf{C}_l^m$ must be zero. This yields the conditions
\begin{equation}\label{eqn:cond-normal}
\begin{aligned}
    \frac{d(rS_{lm}(r))}{dr}\Big|_{a} = 0, \\ 
    T_{lm}(a) = 0.
\end{aligned}
\end{equation}


\subsection{No strictly normal magnetic field under dynamo condition}

Conditions (\ref{eqn:cond-normal}) are quite strict conditions that apply to every Fourier coefficient of the toroidal and the poloidal components. They appear to be too restrictive when used in combination with the so-called \textit{dynamo condition} (Roberts, Treatise 2015):
\begin{equation}\label{eqn:cond-dynamo}
\begin{aligned}
    \left[\frac{dS_{lm}(r)}{dr} + \frac{l+1}{r} S_{lm}(r)\right]_{r=a} &= 0,\\
    % = \frac{dS_{lm}(r)}{dr}\Big|_{a} + \frac{l+1}{a} S_{lm}(a) &= 0 \\ 
    T_{lm}(a) &= 0.
\end{aligned}
\end{equation}
These conditions are the sufficient and necessary conditions that the magnetic field at the boundary can be matched (via continuity of the normal component) to a (quasi-)static magnetic field in electrically insulating medium.
While the condition for the toroidal coefficient coincides with that in Eq. (\ref{eqn:cond-normal}), the condition for the poloidal coefficient does not. Expanding the poloidal condition from (\ref{eqn:cond-normal}) and collecting the poloidal condition from (\ref{eqn:cond-dynamo}), we have
\begin{equation}
    \begin{aligned}
        \frac{dS_{lm}(r)}{dr}\Big|_{a} + \frac{1}{a} S_{lm}(a) &= 0, \\
        \frac{dS_{lm}(r)}{dr}\Big|_{a} + \frac{l+1}{a} S_{lm}(a) &= 0.
    \end{aligned}
\end{equation}
We therefore conclude that we must have (i) $l=0$ and $S'_{lm}(a) + S_{lm}(a)/a = 0$, or (ii) $S_{lm}(a) = S'_{lm}(a) = 0$.
In case (i), the spherical harmonic $Y_{0}^0(\theta, \phi) = 1$, and has only trivial eigenvalue $0$ under the operator $\nabla_s^2$. In case (ii), we have $S_{lm}(a) = 0$. Either way, we have $l(l+1) S_{lm}(r)/r = 0$ at $r=a$. Therefore, not only do the tangent components vanish at the boundary, but the normal component, given by
\[
    B_r = \sum_{l,m} \frac{l(l+1)}{r} S_{lm}(r) Y_l^m(\theta, \phi)
\]
vanishes as well. Therefore, all we have is 
\begin{equation}
    \mathbf{B}|_{r=a} = \mathbf{0}.
\end{equation}
There is no such thing as "normal" because there is no non-trivial normal component! We conclude that there is no magnetic field that is normal at the boundary under dynamo condition.

The question now is that how much we need to relax the conditions for a legitimate magnetic field. It is probably unwise to relax the dynamo condition, as it is a physical contraint that should be fulfilled once we make the assumption that the mantle is electrically insulating. The boundary normal condition is merely a goal we want to achieve, and can of course be flexible.


\section{Almost boundary normal magnetic field}


\end{document}
