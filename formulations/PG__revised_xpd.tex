\section{Challenges in implementing the regularity conditions}


In her dissertation, Daria briefly described how to implement coupling between lowest order coefficients in $s$, namely by expanding some quantities starting from $n=1$, but adding manually the $n=0$ contribution as a coupling term. This implementation (as confirmed by her Mathematica notebook) is equivalent to the following expansion:
\begin{equation}
\begin{aligned}
    B_{es} &= \sum_{n=0}^\infty C_{es}^{mn} \left[(1-s^2) s^{|m|-1} P_n^{(\alpha, \beta)}(2s^2 - 1)\right] \\ 
    B_{e\phi} &= i\sgn(m)\sum_{n=0}^\infty C_{es}^{mn} \left[(1-s^2) s^{|m|-1} P_n^{(\alpha, \beta)}(2s^2 - 1)\right] + \sum_{n=0}^\infty C_{e\phi}^{mn} \left[(1-s^2) s^{|m|+1} P_n^{(\alpha', \beta')}(2s^2 - 1)\right]
\end{aligned}
\end{equation}
for $|m|\geq 1$ (for $m=0$ there is no coupling between these coefficients). 
Note that the bases corresponding to coefficients $C_{e\phi}^{mn}$ has the prefactor $s^{|m|+1}$ instead of $s^{|m|-1}$. 
The Jacobi polynomial indices $\alpha$ and $\beta$ can be chosen relatively freely, so as to enforce maximal sparsity on the matrices. 
For instance, if one uses $(1 - s^2) s^{|m|-1}P_n^{(\alpha, \beta)}(2s^2 - 1)$ and $(1 - s^2) s^{|m|+1}P_n^{(\alpha', \beta')}(2s^2 - 1)$ respectively as the test functions for $B_{es}$ and $B_{e\phi}$ induction equations, a reasonable choice of the indices will be
\[
    \left\{\begin{aligned}
        \alpha &= 2 \\ 
        \beta &= |m| - \frac{3}{2}
    \end{aligned}\right.,\qquad 
    \left\{\begin{aligned}
        \alpha' &= 2 \\
        \beta' &= |m| + \frac{1}{2}
    \end{aligned}\right.
\]
This configuration will diagonalize the matrix blocks $(B_{es}, C_{es}^{mn})$ and $(B_{e\phi}, C_{e\phi}^{mn})$, which are the diagonal blocks in the mass matrix. By $(B, C)$ I denote the matrix block formed by taking the inner product of the test function corresponding to field $B$ and the bases corresponding to coefficient $C$. 
However, the coupling block, i.e. $(B_{e\phi}, C_{es}^{mn})$ will not be diagonal. This will form a dense matrix as an off-diagonal block in the mass matrix.

Although the previous expansion is only for $B_{es}$, $B_{e\phi}$, similar trick can also be used to implement the lowest-order coupling for $B_{es, z}$-$B_{e\phi,z}$ pair, $\widetilde{M_{sz}}$-$\widetilde{M_{\phi z}}$ pairs - basically any quantity pairs that behave like $(s,\phi)$ equatorial components of a vector. 
Daria even applied the same method to implementing the low-order coupling of the rank-$2$ tensors. 
Her implementation (see, e.g. \texttt{C1QP\_reg\_diff\_visc\_daria.nb}) is equivalent to the following expansion:
\[\begin{aligned}
    \overline{M_{s\phi}}^m &= \sum_n C_{s\phi}^{mn} \left[\sqrt{1-s^2}s^{|m|-2}P_n^{(\alpha, \beta)}(2s^2-1)\right] \\ 
    \overline{M_{\phi\phi}}^m &= i\sgn(m)\sum_n C_{s\phi}^{mn} \left[\sqrt{1-s^2}s^{|m|-2}P_n^{(\alpha, \beta)}(2s^2-1)\right] + \sum_n C_{\phi\phi}^{mn} \left[\sqrt{1-s^2}s^{|m|}P_n^{(\alpha', \beta')}(2s^2-1)\right] \\ 
    \widetilde{zM_{s\phi}}^m &= \sum_n C_{zs\phi}^{mn} \left[(1-s^2) s^{|m|-2}P_n^{(\alpha, \beta)}(2s^2-1)\right] \\ 
    \widetilde{zM_{\phi\phi}}^m &= i\sgn(m)\sum_n C_{zs\phi}^{mn} \left[(1-s^2)s^{|m|-2}P_n^{(\alpha, \beta)}(2s^2-1)\right] + \sum_n C_{z\phi\phi}^{mn} \left[{(1-s^2)}s^{|m|}P_n^{(\alpha', \beta')}(2s^2-1)\right]
\end{aligned}\]
Once again, we are looking at $m\geq 2$, as the coupling for $m=0$ is different, and the coupling for $m=\pm 1$ is absent.
This would have worked, had the coupling between tensorial elements in cylindrical coordinates only occurred in the lowest order $n=0$.
However, the previous section has already shown otherwise. In addition to the coupling in $n=0$, we have additional three-component coupling in order $n=1$ for $m\geq 2$, as well as a previously ignored three-component coupling in order $n=0$ for $m=\pm 1$.
Even if the same trick can be used for implementing the three-component coupling in order $n=0$ for $m=\pm 1$:
\[
    2M_{s\phi}^{m0} = i\sgn(m) \left(M_{ss}^{m0} - A_{\phi\phi}^{m0}\right)
\]
by taking the following expansion,
\[\begin{aligned}
    \overline{M_{ss}}^m &= \sum_n C_{ss}^{mn} \left[\sqrt{1-s^2}s P_n^{(\alpha, \beta)}(2s^2-1)\right] \\
    \overline{M_{\phi\phi}}^m &= \sum_n C_{\phi\phi}^{mn} \left[\sqrt{1-s^2}s P_n^{(\alpha, \beta)}(2s^2-1)\right] \\
    \overline{M_{s\phi}}^m &= \frac{i\sgn(m)}{2} \left\{\sum_n C_{ss}^{mn} \left[\sqrt{1-s^2}s P_n^{(\alpha, \beta)}(2s^2-1)\right] - \sum_n C_{\phi\phi}^{mn} \left[\sqrt{1-s^2}s P_n^{(\alpha, \beta)}(2s^2-1)\right]\right\} \\ 
    &\quad + \sum_n C_{s\phi}^{mn} \left[\sqrt{1 - s^2} s^3 P_n^{(\alpha', \beta')}(2s^2 - 1)\right],
\end{aligned}\]
the coupling in order $n=1$ is just not feasible to be implemented in this way. Granted, it may be possible to write down the following expansion,
\[\begin{aligned}
    \overline{M_{ss}}^m &= s^{|m|-2}\sqrt{1-s^2}\left\{C_0 + C_1 s^2 + s^4\sum_n C_{ss}^{mn} \left[P_n^{(\alpha, \beta)}(2s^2-1)\right]\right\} \\
    \overline{M_{\phi\phi}}^m &= s^{|m|-2}\sqrt{1-s^2}\left\{-C_0 + C_2 s^2 + s^4\sum_n C_{\phi\phi}^{mn} \left[P_n^{(\alpha, \beta)}(2s^2-1)\right]\right\} \\
    \overline{M_{s\phi}}^m &= s^{|m|-2}\sqrt{1-s^2}\left\{i\sgn(m)C_0 + \frac{i\sgn(m)}{2}(C_1 - C_2) s^2 + s^4\sum_n C_{s\phi}^{mn} \left[P_n^{(\alpha, \beta)}(2s^2-1)\right]\right\},
\end{aligned}\]
but it then becomes a painstaking task to look for appropriate test functions. 
Note that now we have three additional bases $s^{|m|-2}\sqrt{1 - s^2}$ and $s^{|m|}\sqrt{1 - s^2}$ (occuring twice) in addition to the bases $s^{|m|+2} \sqrt{1 - s^2} P_n^{(\alpha, \beta)}(2s^2 - 1)$. 
Therefore, a total number of $3N+3$ test functions are needed to form a linear system. 
Where exactly do we place the extra three test functions? It seems it doesn't make sense either way and always breaks the symmetry of the problem. 
This difficulty calls for a new expansion of the fields.

