
\section{Why does the quadrature works in SciPy?}

This section is regarding a technical detail on the behaviour of \texttt{eval\_jacobi} in the scipy package, especially when the degree $n$ is negative.
\vspace{1em}

\noindent \textit{Why are there negative degrees in the Jacobi polynomials used in the code?}

In computing the system matrices, stiffness matrix $\mathbf{K}$ in particular, it is often the case that we need to compute the inner product in the form of
\[
    \left\langle s^{m_1}(1-s^2)^{m_2}P_{n'}^{(\alpha', \beta')}(2s^2 - 1), \mathcal{L} \left(s^{m_3}(1-s^2)^{m_4}P_n^{(\alpha, \beta)}(2s^2 - 1)\right) \right\rangle.
\]
The result of the trial function being operated on by the linear operator $\mathcal{L}$ typically involves $P_n^{(\alpha, \beta)}(2s^2-1)$, its derivative with respect to s, i.e. $\frac{d^k}{ds^k} P_n^{(\alpha,\beta)}(2s^2 - 1)$, or in other words, involving derivatives of the Jacobi polynomial $\frac{d^k}{d\xi^k} P_n^{(\alpha, \beta)}(\xi)|_{\xi = 2s^2 - 1}$.
This is not a problem in symbolic engines (actually, the only functioning way in \texttt{SymPy} to calculate such inner products is to keep the derivative as it is, unevaluated), but not acceptable for numerical routines. Typical numerical routines, whether using \texttt{SciPy} in Python or \texttt{MATLAB}, have no idea how to calculate the "derivative of a Jacobi polynomial".
In order to use this numerical routines, the only remaining feasible way seems to be simplifying the expression into explicit polynomials at each given $n$ and $n'$ (using some symbolic engines), and then hand over the explicit polynomial to \texttt{SciPy}.
However, this means that the most desirable feature of this numerical libraries, i.e. vectorized and parallelized operations, are out of the picture.
Evaluating these inner products purely numerically thus seems to encounter a problem.

There is, however, a robust workaround: the derivatives of Jacobi polynomials can always be converted to another Jacobi polynomial, using the relation
\[
    \frac{d^k}{dz^k} P_n^{(\alpha,\beta)}(z) = \frac{\Gamma(\alpha + \beta + n + 1 + k)}{2^k \Gamma(\alpha + \beta + n + 1)} P_{n-k}^{(\alpha+k,\beta+k)}(z).
\]
This can be easily done as soon as \texttt{SymPy} is asked to simplify or "evaluate" the derivatives concerning the Jacobi polynomials.
Now the integrand can be safely converted to a series of algebraic calculations involving only the undifferentiated Jacobi polynomials.
This expression can be handed over to numerical functions, that can be evaluated at multiple $n$, $n'$ as well as $z$ in a vectorized fashion very efficiently. 
However, here comes another question: what is $P_{n-k}^{(\alpha+k,\beta+k)}(z)$, when $n < k$? How will the numerical routine handle this?
\vspace{1em}

\noindent\textit{Is there a Jacobi polynomial with negative degree?}

Strictly/semantically speaking, there is no such a thing.
A polynomial is really just a polynomial, and can only have non-negative degrees.
In fact, if you ask \texttt{SymPy} to evaluate a Jacobi polynomial with negative degree:
\begin{lstlisting}[language=Python]
>>> sympy.jacobi(-1, 5/2, 4, -0.9).evalf()
...
ValueError: Cannot generate Jacobi polynomial of degree -1
\end{lstlisting}
Indeed, if we follow the definition on Wikipedia page,
\[
    P_n^{(\alpha,\beta)}(z) = \frac{(\alpha + 1)_n}{n!}\prescript{}{2}{F}_1\left(-n, 1+\alpha+\beta+n, 1+\alpha, \frac{1-z}{2}\right)
\]
where $\prescript{}{2}{F}_1$ is the hypergeometric function.
The form of the prefactor is apparently only restricted to non-negative $n$, since factorial as well as Pochhammer's symbol usually only takes non-negative $n$ arguments.
However, both \texttt{Mathematica} and \texttt{SciPy} are okay with evaluating Jacobi polynomials with negative degrees:
\begin{lstlisting}[language=Mathematica]
(*Mathematica*)
In[1]=  N[JacobiP[-1, 5/2, 4, -0.9]]
Out[1]= 0.
\end{lstlisting}
\begin{lstlisting}[language=Python]
# Python
>>> from scipy.special import eval_jacobi
>>> eval_jacobi(-1, 5/2, 4, -0.9)
0.
\end{lstlisting}
As long as... well, the polynomial with the negative degree is not evaluated at one specific point, $z=-1$:
\begin{lstlisting}[language=Mathematica]
(*Mathematica*)
In[1]=  N[JacobiP[-1, 5/2, 4, -1]]
   ... Power: Infinite expression 1/0^4 encountered
   ... Infinity: Indeterminate expression 0 ComplexInfinity encountered
Out[1]= Indeterminate
\end{lstlisting}
\begin{lstlisting}[language=Python]
# Python
>>> from scipy.special import eval_jacobi
>>> eval_jacobi(-1, 5/2, 4, -1.)
nan
\end{lstlisting}
But why is this the case, if the polynomial shouldn't even have negative degree?
\vspace{1em}

\noindent\textit{What is the implementation for the Jacobi polynomial with negative degree?}

Now we have to understand what is happening behind the curtain: how is the Jacobi polynomial actually implemented, such that both Mathematica and \texttt{SciPy} allow negative degrees?
Will this guarantee that the quadrature of the inner product is correct?
The best way to answer these questions is to check the source code.
Unfortunately, this does not work for \texttt{Mathematica}, a closed-source software.
This can however be done for \texttt{SciPy}.
It took me a while to find the relevant piece in the source code, as this part is in Cython:
\begin{lstlisting}[language=Python]
cdef inline number_t eval_jacobi(double n, double alpha, double beta, number_t x) noexcept nogil:
    cdef double a, b, c, d
    cdef number_t g

    d = binom(n+alpha, n)
    a = -n
    b = n + alpha + beta + 1
    c = alpha + 1
    g = 0.5*(1-x)
    return d * hyp2f1(a, b, c, g)
\end{lstlisting}
It turns out that instead of using the factorial and Pochhammer's symbol, \texttt{SciPy} implements the following relation
\[
    P_n^{(\alpha,\beta)}(z) = \begin{pmatrix}n+\alpha \\ n\end{pmatrix}\prescript{}{2}{F}_1\left(-n, 1+\alpha+\beta+n, 1+\alpha, \frac{1-z}{2}\right)
\]
where the prefactor is given by a binomial coefficient. Usually, the binomial coefficient does not make sense for negative $n$, but if one further looks at the source code for \texttt{binom}, one would realize that except for special occasions, the binomial coefficients is calculated as
\[
    \begin{pmatrix} n + \alpha \\ n \end{pmatrix} = \frac{1}{(n + \alpha + 1)B(1 + \alpha, 1 + n)} = \frac{1}{n + \alpha + 1} \frac{\Gamma(2 + \alpha + n)}{\Gamma(1 + \alpha) \Gamma(1 + n)}
\]
which gives $0$ for any $n\in \mathbb{Z}^- \cup \{0\}$ when $\alpha + n \notin \mathbb{Z}^-$ (because $\Gamma(1+n)\rightarrow \infty$). Now, the second criterion is always fulfilled. As $P_{n-k}^{(\alpha + k, \beta + k)}$ is the ultimate function to be evaluated, our $n+\alpha$ is actually $(n-k)+(\alpha +k) = n+\alpha$. Since the original Jacobi polynomial is a legitimate polynomial, $\alpha > -1$ and $n\geq 0$, therefore $n+\alpha \notin \mathbb{Z}^-$. In summary, \texttt{SciPy} will give $0$, which is the desired outcome, when a Jacobi polynomial with negative degree is encountered.

On a related note, the Jacobi polynomial in the form of
\begin{equation}
    P_n^{(\alpha,\beta)}(z) = \frac{1}{n + \alpha + 1} \frac{\Gamma(2 + \alpha + n)}{\Gamma(1 + \alpha) \Gamma(1 + n)} \prescript{}{2}{F}_1\left(-n, 1+\alpha+\beta+n, 1+\alpha, \frac{1-z}{2}\right)
\end{equation}
might be a good formula for analytic continuation in $\alpha$, $\beta$, $n$. This formula seems to have finite value for any point, except for a zero-measure set in the 4-D space.

So why does the evaluation fail at $z=-1$ for negative $n$? This is the branch point for the hypergeometric function at $n\in \mathbb{Z}^-$, and evaluation is not available even for the equation above.
