\section{New expansion for coupled quantities}


Is there a way to circumvent manually enforcing all of these regularity constraints by designing intricate expansions?
The answer is yes, according to Matthew and Stefano.
For the vector quantities in cylindrical coordinates, i.e. components $A_s$ and $A_\phi$, they suggest that instead of expanding them separately, one should rather be looking for expansions of $A_s \pm i A_\phi$. These will have the expansion
\[\begin{aligned}
    A_s + iA_\phi &= \sum_m s^{|m+1|} p(s^2) e^{im\phi} \\ 
    A_s - iA_\phi &= \sum_m s^{|m-1|} p(s^2) e^{im\phi}
\end{aligned}\]
These are, if I may, conjugate quantities to the original components, whose regularity is the sufficient and necessary condition that the corresponding Cartesian components are regular.

Does this trick similarly apply to rank-2 tensors?
Indeed, if we take a step back from the final regularity constraints on individual matrix elements in cylindrical coordinates, we find that as an intermediate step, we have eq.(\ref{eqn:regularity-constraint-tensor-all}), a set of regularity constraints on the Fourier coefficients that are sufficient and necessary conditions that the corresponding Cartesian components have regular Fourier coefficients.
These are, after all, what gave rise to the regularity constraints on individual variables.
It follows directly, that the components in the cylindrical coordinates need to and only need to have the following expansion
\begin{equation}
    \begin{aligned}
        A_{ss} + A_{\phi\phi} &= \sum_{m} s^{|m|} C(s^2) e^{im\phi} \\ 
        A_{s\phi} - A_{\phi s} &= \sum_{m} s^{|m|} C(s^2) e^{im\phi} \\ 
        A_{ss} - A_{\phi\phi} + i \left(A_{s\phi} + A_{\phi s}\right) &= \sum_{m} s^{|m + 2|} C(s^2) e^{im\phi} \\ 
        A_{ss} - A_{\phi\phi} - i \left(A_{s\phi} + A_{\phi s}\right) &= \sum_{m} s^{|m - 2|} C(s^2) e^{im\phi}.
    \end{aligned}
\end{equation}

It is further beneficial to restrict the discussion to the relevant scenario at hand: the tensor is formed by the outer product of a vector with itself, and is of course symmetric.
In this case the four relations are reduced to three relations regarding three quantities
\begin{equation}
    \begin{aligned}
        M_{ss} + M_{\phi\phi} = M_1 &= \sum_{m} s^{|m|} C(s^2) e^{im\phi} \\ 
        M_{ss} - M_{\phi\phi} + i 2M_{s\phi} = M_+ &= \sum_{m} s^{|m + 2|} C(s^2) e^{im\phi} \\ 
        M_{ss} - M_{\phi\phi} - i 2M_{s\phi} = M_- &= \sum_{m} s^{|m - 2|} C(s^2) e^{im\phi}.
    \end{aligned}
\end{equation}
Here we introduced symbols for these conjugate quantities, i.e. $M_1$, $M_+$ and $M_-$. 
Note that among these quantities, only $M_1$ is strictly a scalar quantity, as only $M_1$ exhibits true scalar regularity condition (i.e. $\sim s^{|m|}$). 
Indeed, a careful reader may have realized by now that $M_1$ is nothing but the trace of the 2-D tensor $\mathbf{M}$. 
That $\mathrm{Tr}\mathbf{M}$ remains a constant during rotation of axes is a known fact, hence it is also called "the first invariant", especially in the context of strain and stress.

What about the other two quantities? In the context of an outer product tensor $\mathbf{M} = \mathbf{B} \mathbf{B}^T$, one way to look at it is that the other two conjugate quantities are the squares of the \textit{vector} conjugate quantities:
\[
    M_{ss} - M_{\phi\phi} \pm i2 M_{s\phi} = \left(B_s \pm i B_\phi\right)^2
\]
There doesn't seem to be any further immediate physical meaning attached to these quantities.

It is useful to observe that these conjugate quantities are linked to the cylindrical components via an invertible linear map, that is independent of $\phi$,
\begin{equation}\label{eqn:linmap-conjugate}
    \begin{pmatrix} M_1 \\ M_+ \\ M_- \end{pmatrix} = 
    \begin{pmatrix}
        1 & 1 & 0 \\
        1 & -1 & 2i \\ 
        1 & -1 & - 2i
    \end{pmatrix}
    \begin{pmatrix} M_{ss} \\ M_{\phi\phi} \\ M_{s\phi} \end{pmatrix},\qquad 
    \begin{pmatrix} M_{ss} \\ M_{\phi\phi} \\ M_{s\phi} \end{pmatrix} = \frac{1}{4}
    \begin{pmatrix}
        2 & 1 & 1 \\
        2 & -1 & -1 \\ 
        0 & -i & i
    \end{pmatrix}
    \begin{pmatrix} M_1 \\ M_+ \\ M_- \end{pmatrix}.
\end{equation}
\medskip

Before we move on, we note that all of these conjugate quantities have a common status.
If we first look at the transform for the vector components in cylindrical coordinates
\[
    \begin{pmatrix} B_x \\ B_y \end{pmatrix} = 
    \begin{pmatrix}
        \cos\phi & -\sin\phi \\ 
        \sin\phi & \cos\phi 
    \end{pmatrix} \begin{pmatrix} B_s \\ B_\phi \end{pmatrix} = \frac{1}{2}
    \begin{pmatrix}
        e^{i\phi} + e^{-i\phi} & i \left(e^{i\phi} - e^{-i\phi}\right) \\ 
        - i \left(e^{i\phi} - e^{-i\phi}\right) & e^{i\phi} + e^{-i\phi}
    \end{pmatrix} \begin{pmatrix} B_s \\ B_\phi \end{pmatrix} = \mathbf{R} \begin{pmatrix} B_s \\ B_\phi \end{pmatrix}
\]
The rotation matrix has the following spectral decomposition,
\[
    \mathbf{R} =
    \begin{pmatrix} 1 & 1 \\ -i & i \end{pmatrix}
    \begin{pmatrix} e^{i\phi} & 0 \\ 0 & e^{-i\phi} \end{pmatrix}
    \begin{pmatrix} 1 & 1 \\ -i & i \end{pmatrix}^{-1} = \frac{1}{2}
    \begin{pmatrix} 1 & 1 \\ -i & i \end{pmatrix}
    \begin{pmatrix} e^{i\phi} & 0 \\ 0 & e^{-i\phi} \end{pmatrix}
    \begin{pmatrix} 1 & i \\ 1 & -i \end{pmatrix}
\]
i.e. eigenvector $(1, \pm i)$ corresponding to eigenvalues $e^{\mp i\phi}$. This means that certain linear combinations (given by the \textit{inverse} of the eigenvalue matrix) of the components retain their form during rotation, except for an additional phase factor:
\[\begin{aligned}
    B_x + iB_y &= e^{+i\phi} \left(B_s + iB_\phi\right) \\ 
    B_x - iB_y &= e^{-i\phi} \left(B_s - iB_\phi\right)
\end{aligned} \quad \Longrightarrow\quad 
\begin{aligned}
    B_s + iB_\phi &= e^{-i\phi} \left(B_x + iB_y\right) \\ 
    B_s - iB_\phi &= e^{+i\phi} \left(B_x - iB_y\right).
\end{aligned}\]
Therefore, the conjugate quantities are nothing but from the (inverse of the) eigenvectors of the rotation matrix.
Moreover, one can immediately deduce the regular expansion from these relations.
We know the Cartesian components behave like scalars, so the right-hand-sides have Fourier coefficients that are $\sim s^{|m|}$.
Therefore, the valid expansion for these conjugate quantities would be
\[\begin{aligned}
    B_s + iB_\phi = \sum_m s^{|m+1|}p(s^2) e^{im\phi} \\ 
    B_s - iB_\phi = \sum_m s^{|m-1|}p(s^2) e^{im\phi}
\end{aligned}\]
exactly as we expected.
Similarly, let us consider the transform of rank-2 tensors in the form of matrices between cylindrical and Cartesian coordinates.
It can of course be done via
\[\begin{pmatrix} M_{xx} & M_{xy} \\ M_{yx} & M_{yy} \end{pmatrix} = 
\begin{pmatrix} \cos\phi & -\sin\phi \\ \sin\phi & \cos\phi \end{pmatrix}  
\begin{pmatrix} M_{ss} & M_{s\phi} \\ M_{\phi s} & M_{\phi \phi} \end{pmatrix}
\begin{pmatrix} \cos\phi & \sin\phi \\ -\sin\phi & \cos\phi \end{pmatrix}
\]
but since this is after all just a linear transform, the whole operation can be written in matrix-vector multiplication if we flatten the tensor into vectors,
\[\begin{pmatrix} M_{xx} \\ M_{yy} \\ M_{xy} \end{pmatrix} = 
\begin{pmatrix}
    \cos^2\phi & \sin^2\phi & -2\sin\phi\cos\phi \\
    \sin^2\phi & \cos^2\phi & +2\sin\phi\cos\phi \\
    \cos\phi \sin\phi & -\cos\phi \sin\phi & \cos^2\phi - \sin^2\phi
\end{pmatrix}
\begin{pmatrix} M_{ss} \\ M_{\phi\phi} \\ M_{s\phi} \end{pmatrix} 
= \mathbf{R}_2 \begin{pmatrix} M_{ss} \\ M_{\phi\phi} \\ M_{s\phi} \end{pmatrix}.
\]
Here we already assumes that the tensor is symmetric. The augmented rotation matrix for the rank-2 tensor can be shown to have eigendecomposition
\[\begin{aligned}
    \mathbf{R}_2 &=
    \begin{pmatrix}
        1 & 1 & 1\\
        1 & -1 & -1\\
        0 & -i & i
    \end{pmatrix}
    \begin{pmatrix} 1 & & \\ & e^{i2\phi} & \\ & & e^{-i2\phi} \end{pmatrix}
    \begin{pmatrix}
        1 & 1 & 1\\
        1 & -1 & -1\\
        0 & -i & i
    \end{pmatrix}^{-1} \\ 
    &= \frac{1}{4} \begin{pmatrix}
        1 & 1 & 1\\
        1 & -1 & -1\\
        0 & -i & i
    \end{pmatrix}
    \begin{pmatrix} 1 & & \\ & e^{i2\phi} & \\ & & e^{-i2\phi} \end{pmatrix}
    \begin{pmatrix}
        2 & 2 & 0\\
        1 & -1 & 2i\\
        1 & -1 & -2i
    \end{pmatrix}
\end{aligned}\]
Note in this case the rotation matrix $\mathbf{R}_2$ is not diagonalized by a Hermitian matrix, and so the inverse of the eigenvector matrix is different from the complex conjugate transpose of the original eigenvector matrix.
This gives the three relations that we have already obtained before,
\[\begin{aligned}
    M_{xx} + M_{yy} &= M_{ss} + M_{s\phi} \\ 
    M_{xx} - M_{yy} + 2i M_{xy} &= e^{i2\phi} \left(M_{ss} - M_{\phi\phi} + 2i M_{s\phi}\right) \\ 
    M_{xx} - M_{yy} - 2i M_{xy} &= e^{-i2\phi} \left(M_{ss} - M_{\phi\phi} - 2i M_{s\phi}\right)
\end{aligned}\]
which will give the regularity conditions given the scalar property of Cartesian components.

To summarize, all of these conjugate quantities, at least for rank-1 and rank-2 tensors, can be derived by computing the eigenvalue decomposition of the rotation matrix for the flattened component vector,
\[\mathbf{R}_k = \mathbf{V} \bm{\Lambda} \mathbf{V}^{-1}\]
and so the relation
\[\mathbf{y}^{\mathrm{Cart}} = \mathbf{R}_k \mathbf{y}^{\mathrm{Cyl}} \quad \Longrightarrow \quad \mathbf{V}^{-1} \mathbf{y}^{\mathrm{Cart}} = \bm{\Lambda} \left(\mathbf{V}^{-1} \mathbf{y}^{\mathrm{Cyl}}\right)
\]
gives the conjugate quantities that happen to retain their forms during changing coordinates systems.
However, it is doubted whether such expression can be found for arbitrary rank tensor.
For an arbitrary rank tensor in general, it would suffice to seek the matrix decomposition of the rotation matrix in the form of
\begin{equation}
    \mathbf{R}_k = \mathbf{V} \bm{\Lambda} \mathbf{U}
\end{equation}
where $\mathbf{V}, \mathbf{U} \in \mathbb{C}^{2^k\times 2^k}$ are invertible matrices whose elements are constants independent of $\phi$, and $\bm{\Lambda} = \mathrm{diag}\left(C_j e^{i n_j \phi}\right)$ is a diagonal matrix whose diagonal entries are solely an exponential function of $\phi$.
If such a factorization is found, the rotation transform can be rewritten as
\begin{equation}
    \mathbf{V}^{-1} \mathbf{y}^\mathrm{Cart} = \bm{\Lambda} \left(\mathbf{U} \mathbf{y}^\mathrm{Cyl}\right)
\end{equation}
and the elements in $\mathbf{U} \mathbf{y}^\mathrm{Cyl}$ would give the $2^k$ \textit{conjugate} quantities whose Fourier coefficients take the form of $s^{|m+n_j|\phi}$.


\section{Evolution equation for conjugate quantities}

As previously implied, a natural way to circumvent manually enforcing all regularity constraints is to instead expand the conjugate quantities.
For the 12 magnetic quantities (except for $B_r$, which lives on the sphere, and $B_{ez}$, which is a scalar) in PG variables, the corresponding conjugate quantities are
\begin{equation}\label{eqn:conjugate-transform}
\begin{aligned}
    &\left\{\begin{aligned}
        \overline{M_{ss}} \\ 
        \overline{M_{\phi\phi}} \\ 
        \overline{M_{s\phi}} \\ 
    \end{aligned}\right. &\longrightarrow \quad 
    &\left\{\begin{aligned}
        \overline{M_1} &= \overline{M_{ss}} + \overline{M_{\phi\phi}} \\ 
        \overline{M_+} &= \overline{M_{ss}} - \overline{M_{\phi\phi}} + i2 \overline{M_{s\phi}} \\ 
        \overline{M_-} &= \overline{M_{ss}} - \overline{M_{\phi\phi}} - i2 \overline{M_{s\phi}}
    \end{aligned}\right. &\quad
    &\left\{\begin{aligned}
        \overline{M_{ss}} &= \frac{1}{2} \overline{M_1} + \frac{1}{4}\overline{M_+} + \frac{1}{4}\overline{M_-} \\ 
        \overline{M_{\phi\phi}} &= \frac{1}{2}\overline{M_1} - \frac{1}{4} \overline{M_+} - \frac{1}{4} \overline{M_-} \\ 
        \overline{M_{s\phi}} &= -\frac{i}{4}\overline{M_+} + \frac{i}{4} \overline{M_-}
    \end{aligned}\right. \\ 
    &\left\{\begin{aligned}
        \widetilde{M_{sz}} \\ 
        \widetilde{M_{\phi z}} \\ 
    \end{aligned}\right. &\longrightarrow \quad 
    &\left\{\begin{aligned}
        \widetilde{M_{z+}} &= \widetilde{M_{sz}} + i\widetilde{M_{\phi z}} \\ 
        \widetilde{M_{z-}} &= \widetilde{M_{sz}} - i\widetilde{M_{\phi z}}
    \end{aligned}\right. &\quad 
    &\left\{\begin{aligned}
        \widetilde{M_{sz}} &= \frac{1}{2} \widetilde{M_{z+}} + \frac{1}{2} \widetilde{M_{z-}} \\
        \widetilde{M_{\phi z}} &= -\frac{i}{2} \widetilde{M_{z+}} + \frac{i}{2} \widetilde{M_{z-}}
    \end{aligned}\right.\\
    &\left\{\begin{aligned}
        \widetilde{zM_{ss}} \\ 
        \widetilde{zM_{\phi\phi}} \\ 
        \widetilde{zM_{s\phi}} \\ 
    \end{aligned}\right. &\longrightarrow \quad 
    &\left\{\begin{aligned}
        \widetilde{zM_1} &= \widetilde{zM_{ss}} + \widetilde{zM_{\phi\phi}} \\ 
        \widetilde{zM_+} &= \widetilde{zM_{ss}} - \widetilde{zM_{\phi\phi}} + i2 \widetilde{zM_{s\phi}} \\ 
        \widetilde{zM_-} &= \widetilde{zM_{ss}} - \widetilde{zM_{\phi\phi}} - i2 \widetilde{zM_{s\phi}}
    \end{aligned}\right. &\quad
    &\left\{\begin{aligned}
        \widetilde{zM_{ss}} &= \frac{1}{2} \widetilde{zM_1} + \frac{1}{4}\widetilde{zM_+} + \frac{1}{4}\widetilde{zM_-} \\ 
        \widetilde{zM_{\phi\phi}} &= \frac{1}{2}\widetilde{zM_1} - \frac{1}{4} \widetilde{zM_+} - \frac{1}{4} \widetilde{zM_-} \\ 
        \widetilde{zM_{s\phi}} &= -\frac{i}{4}\widetilde{zM_+} + \frac{i}{4} \widetilde{zM_-}
    \end{aligned}\right. \\
    &\left\{\begin{aligned}
        B_{es} \\ 
        B_{e\phi}
    \end{aligned}\right. &\longrightarrow \quad 
    &\left\{\begin{aligned}
        B_{e+} &= B_{es} + iB_{e\phi} \\ 
        B_{e-} &= B_{es} - iB_{e\phi}
    \end{aligned}\right. &\quad 
    &\left\{\begin{aligned}
        B_{es} &= \frac{1}{2} B_{e+} + \frac{1}{2} B_{e-} \\
        B_{e\phi} &= -\frac{i}{2} B_{e+} + \frac{i}{2} B_{e-}
    \end{aligned}\right.\\
    &\left\{\begin{aligned}
        B_{es, z} \\ 
        B_{e\phi, z}
    \end{aligned}\right. &\longrightarrow \quad 
    &\left\{\begin{aligned}
        B_{e+,z} &= B_{es,z} + iB_{e\phi,z} \\ 
        B_{e-,z} &= B_{es,z} - iB_{e\phi,z}
    \end{aligned}\right. &\quad 
    &\left\{\begin{aligned}
        B_{es,z} &= \frac{1}{2} B_{e+,z} + \frac{1}{2} B_{e-,z}\\
        B_{e\phi, z} &= -\frac{i}{2} B_{e+,z} + \frac{i}{2} B_{e-,z}
    \end{aligned}\right.
\end{aligned}\end{equation}
% The inverse map of these quantities are calculated via eq.(\ref{eqn:linmap-conjugate}).
These conjugate quantities have very simple regularity constraints.
Their regularity constraint is merely a leading order behaviour in $s$, determined by their scalar Cartesian counterparts, and a leading order behaviour in $H$ from even or odd axial integration.
The Fourier coefficients for these quantities are
\begin{equation}
\begin{aligned}
    \overline{M_1}^m &= H s^{|m|} p(s^2)\\
    \overline{M_+}^m &= H s^{|m+2|} p(s^2)\\
    \overline{M_-}^m &= H s^{|m-2|} p(s^2)\\ 
    \widetilde{M_{z+}}^m &= H^2 s^{|m+1|} p(s^2)\\
    \widetilde{M_{z-}}^m &= H^2 s^{|m-1|} p(s^2)\\
    \widetilde{zM_1}^m &= H^2 s^{|m|} p(s^2)\\
    \widetilde{zM_+}^m &= H^2 s^{|m+2|} p(s^2)\\
    \widetilde{zM_-}^m &= H^2 s^{|m-2|} p(s^2)\\
    B_{e+}^m &= s^{|m+1|} p(s^2)\\
    B_{e-}^m &= s^{|m-1|} p(s^2)\\
    B_{e+,z}^m &= s^{|m+1|} p(s^2)\\
    B_{e-,z}^m &= s^{|m-1|} p(s^2)
\end{aligned}
\end{equation}
where $p(s^2)$ denotes any analytic function in $s^2$. 
When the expansion for the equatorial fields are further combined with a harmonic field contribution, the equatorial Fourier coefficients will have a further $H^2$ prefactor in the front. 
Apart from that, they are free of any coupling in their Fourier coefficients. The leading order behaviour alone guarantees regularity.

Despite all the merits with this set of expansions, it is not directly applicable to the current form of the PG equations. 
The reason lies in the test functions to be used to reduce the equations into linear systems.
With every tensor component comprising of multiple bases, there is no straightforward and consistent way to choose a set of test functions.

One way to overcome the test function issue, and perhaps the most consistent way, is to expand the conjugate quantities in their evolution equations.
In other words, the evolution equations in terms of magnetic field quantities should first be transformed into evolution equations in their conjugate quantities.
I shall present here the explicit derivation of one set of these quantities, namely from $(\overline{M_{ss}}, \overline{M_{s\phi}}, \overline{M_{\phi\phi}})$ to $(\overline{M_1}, \overline{M_+}, \overline{M_-})$.
Starting from the original evolution equations
\[\begin{aligned}
    \frac{\partial \overline{M_{ss}}}{\partial t} &= -H (\mathbf{u}\cdot \nabla_e) \frac{\overline{M_{ss}}}{H} + 2 \overline{M_{ss}} \frac{\partial u_s}{\partial s} + \frac{2}{s} \overline{M_{s\phi}} \frac{\partial u_s}{\partial \phi} \\ 
    \frac{\partial \overline{M_{\phi\phi}}}{\partial t} &= -\frac{1}{H} (\mathbf{u}\cdot \nabla_e) \left(H \overline{M_{\phi\phi}}\right) - 2 \overline{M_{\phi\phi}} \frac{\partial u_s}{\partial s} + 2 s \overline{M_{s\phi}} \frac{\partial}{\partial s}\left(\frac{u_\phi}{s}\right) \\ 
    \frac{\partial \overline{M_{s\phi}}}{\partial t} &= - (\mathbf{u}\cdot \nabla_e) \overline{M_{s\phi}} + s \overline{M_{ss}} \frac{\partial}{\partial s}\left(\frac{u_\phi}{s}\right) + \frac{1}{s} \overline{M_{\phi\phi}} \frac{\partial u_s}{\partial \phi}
\end{aligned}\]
Using the transforms (\ref{eqn:conjugate-transform}), the equations can be rewritten as
\[\begin{aligned}
    \frac{\partial \overline{M_{ss}}}{\partial t} &= - (\mathbf{u}\cdot \nabla_e) \overline{M_{ss}} + \frac{1}{4}\left(2\overline{M_1} + \overline{M_+} + \overline{M_-}\right) \left(2 \frac{\partial u_s}{\partial s} - \frac{su_s}{H^2}\right) - \frac{i}{2} \left(\overline{M_+} - \overline{M_-}\right) \frac{1}{s} \frac{\partial u_s}{\partial \phi} \\ 
    \frac{\partial \overline{M_{\phi\phi}}}{\partial t} &= - (\mathbf{u}\cdot \nabla_e) \overline{M_{\phi\phi}} - \frac{1}{4}\left(2\overline{M_1} - \overline{M_+} - \overline{M_-}\right) \left(2 \frac{\partial u_s}{\partial s} - \frac{su_s}{H^2}\right) - \frac{i}{2} \left(\overline{M_+} - \overline{M_-}\right) s \frac{\partial}{\partial s} \frac{u_\phi}{s} \\ 
    \frac{\partial \overline{M_{s\phi}}}{\partial t} &= - (\mathbf{u}\cdot \nabla_e) \overline{M_{s\phi}} + \frac{1}{4}\left(2\overline{M_1} + \overline{M_+} + \overline{M_-}\right) s \frac{\partial}{\partial s}\left(\frac{u_\phi}{s}\right) + \frac{1}{4}\left(2\overline{M_1} - \overline{M_+} - \overline{M_-}\right) \frac{1}{s} \frac{\partial u_s}{\partial \phi}
\end{aligned}\]
Re-combining these equations again using the transforms (\ref{eqn:conjugate-transform}), we obtain the evolution equations for the conjugate variables
\[\begin{aligned}
    \frac{\partial \overline{M_1}}{\partial t} &= - \left(\mathbf{u}\cdot \nabla_e \right) \overline{M_1} + \left(\overline{M_+} + \overline{M_-}\right) \left(\frac{\partial u_s}{\partial s} - \frac{su_s}{2H^2}\right) - \frac{i}{2} \left(\overline{M_+} - \overline{M_-}\right) \left(s \frac{\partial}{\partial s}\frac{u_\phi}{s} + \frac{1}{s}\frac{\partial u_s}{\partial \phi}\right) \\
    \frac{\partial \overline{M_+}}{\partial t} &= - \left(\mathbf{u}\cdot \nabla_e \right) \overline{M_+} + \overline{M_1} \left(2\frac{\partial u_s}{\partial s} - \frac{su_s}{H^2}\right) + i\overline{M_1} \left(s \frac{\partial}{\partial s}\frac{u_\phi}{s} + \frac{1}{s}\frac{\partial u_s}{\partial \phi}\right) + i \overline{M_+} \left(s \frac{\partial}{\partial s}\frac{u_\phi}{s} - \frac{1}{s}\frac{\partial u_s}{\partial \phi}\right) \\ 
    \frac{\partial \overline{M_-}}{\partial t} &= - \left(\mathbf{u}\cdot \nabla_e \right) \overline{M_-} + \overline{M_1} \left(2\frac{\partial u_s}{\partial s} - \frac{su_s}{H^2}\right) - i\overline{M_1} \left(s \frac{\partial}{\partial s}\frac{u_\phi}{s} + \frac{1}{s}\frac{\partial u_s}{\partial \phi}\right) - i \overline{M_-} \left(s \frac{\partial}{\partial s}\frac{u_\phi}{s} - \frac{1}{s}\frac{\partial u_s}{\partial \phi}\right)
\end{aligned}\]
The complete list of evolution equations for the conjugate quantities (12$\times$) are given by
\begin{align*}
    \frac{\partial \overline{M_1}}{\partial t} &= - \left(\mathbf{u}\cdot \nabla_e \right) \overline{M_1} + \left(\overline{M_+} + \overline{M_-}\right) \left(\frac{\partial u_s}{\partial s} - \frac{su_s}{2H^2}\right) - \frac{i}{2} \left(\overline{M_+} - \overline{M_-}\right) \left(s \frac{\partial}{\partial s}\frac{u_\phi}{s} + \frac{1}{s}\frac{\partial u_s}{\partial \phi}\right) \\
    \frac{\partial \overline{M_+}}{\partial t} &= - \left(\mathbf{u}\cdot \nabla_e \right) \overline{M_+} + \overline{M_1} \left(2\frac{\partial u_s}{\partial s} - \frac{su_s}{H^2}\right) + i\overline{M_1} \left(s \frac{\partial}{\partial s}\frac{u_\phi}{s} + \frac{1}{s}\frac{\partial u_s}{\partial \phi}\right) + i \overline{M_+} \left(s \frac{\partial}{\partial s}\frac{u_\phi}{s} - \frac{1}{s}\frac{\partial u_s}{\partial \phi}\right) \\ 
    \frac{\partial \overline{M_-}}{\partial t} &= - \left(\mathbf{u}\cdot \nabla_e \right) \overline{M_-} + \overline{M_1} \left(2\frac{\partial u_s}{\partial s} - \frac{su_s}{H^2}\right) - i\overline{M_1} \left(s \frac{\partial}{\partial s}\frac{u_\phi}{s} + \frac{1}{s}\frac{\partial u_s}{\partial \phi}\right) - i \overline{M_-} \left(s \frac{\partial}{\partial s}\frac{u_\phi}{s} - \frac{1}{s}\frac{\partial u_s}{\partial \phi}\right) \\ 
    %
    \frac{\partial \widetilde{M_{z+}}}{\partial t} &= - (\mathbf{u}\cdot \nabla_e) \widetilde{M_{z+}} + \frac{1}{2}\left(3 \frac{\partial u_z}{\partial z} - \frac{i}{s}\frac{\partial u_s}{\partial \phi} + is \frac{\partial}{\partial s}\frac{u_\phi}{s}\right) \widetilde{M_{z+}} + \frac{1}{2} \left(\frac{\partial u_z}{\partial z} + 2 \frac{\partial u_s}{\partial s} + \frac{i}{s} \frac{\partial u_s}{\partial \phi} + is \frac{\partial}{\partial s}\frac{u_\phi}{s}\right) \widetilde{M_{z-}} \\
    &\quad -\frac{1}{2} \left(\frac{\partial}{\partial s} \frac{su_s}{H^2} + \frac{i}{H^2} \frac{\partial u_s}{\partial \phi}\right) \widetilde{zM_1} + \frac{1}{2} \left(-\frac{\partial}{\partial s}\frac{su_s}{H^2} + \frac{i}{H^2}\frac{\partial u_s}{\partial \phi}\right) \widetilde{zM_+} \\
    \frac{\partial \widetilde{M_{z-}}}{\partial t} &= - (\mathbf{u}\cdot \nabla_e) \widetilde{M_{z-}} + \frac{1}{2} \left(\frac{\partial u_z}{\partial z} + 2 \frac{\partial u_s}{\partial s} - \frac{i}{s} \frac{\partial u_s}{\partial \phi} - is \frac{\partial}{\partial s}\frac{u_\phi}{s}\right) \widetilde{M_{z+}} + \frac{1}{2}\left(3 \frac{\partial u_z}{\partial z} + \frac{i}{s}\frac{\partial u_s}{\partial \phi} - is \frac{\partial}{\partial s}\frac{u_\phi}{s}\right) \widetilde{M_{z-}} \\
    &\quad -\frac{1}{2} \left(\frac{\partial}{\partial s} \frac{su_s}{H^2} - \frac{i}{H^2} \frac{\partial u_s}{\partial \phi}\right) \widetilde{zM_1} + \frac{1}{2} \left(-\frac{\partial}{\partial s}\frac{su_s}{H^2} - \frac{i}{H^2}\frac{\partial u_s}{\partial \phi}\right) \widetilde{zM_-} \\
    %
    \frac{\partial \widetilde{zM_1}}{\partial t} &= -(\mathbf{u}\cdot \nabla_e) \widetilde{zM_1} + \frac{\partial u_z}{\partial z} \widetilde{zM_1} + \left(\frac{\partial u_s}{\partial s} + \frac{1}{2}\frac{\partial u_z}{\partial z}\right) \left(\widetilde{zM_+} + \widetilde{zM_-}\right) - \frac{i}{2}\left(\frac{1}{s}\frac{\partial u_s}{\partial \phi} + s \frac{\partial}{\partial s}\frac{u_\phi}{s}\right) \left(\widetilde{zM_+} - \widetilde{zM_-}\right) \\
    \frac{\partial \widetilde{zM_+}}{\partial t} &= - (\mathbf{u}\cdot \nabla_e) \widetilde{zM_+} + \frac{\partial u_z}{\partial z} \widetilde{zM_+} + i \left(s\frac{\partial}{\partial s}\frac{u_\phi}{s} - \frac{1}{s}\frac{\partial u_s}{\partial \phi}\right) \widetilde{zM_+} + \left(2\frac{\partial u_s}{\partial s} + \frac{\partial u_z}{\partial z} + is \frac{\partial}{\partial s}\frac{u_\phi}{s}+ \frac{i}{s}\frac{\partial u_s}{\partial \phi}\right) \widetilde{zM_1} \\
    \frac{\partial \widetilde{zM_-}}{\partial t} &= - (\mathbf{u}\cdot \nabla_e) \widetilde{zM_-} + \frac{\partial u_z}{\partial z} \widetilde{zM_-} - i \left(s\frac{\partial}{\partial s}\frac{u_\phi}{s} - \frac{1}{s}\frac{\partial u_s}{\partial \phi}\right) \widetilde{zM_-} + \left(2\frac{\partial u_s}{\partial s} + \frac{\partial u_z}{\partial z} - is \frac{\partial}{\partial s}\frac{u_\phi}{s}- \frac{i}{s}\frac{\partial u_s}{\partial \phi}\right) \widetilde{zM_1} \\
    %
    \frac{\partial B_{e+}}{\partial t} &= - \left(\mathbf{u}_e \cdot \nabla_e\right) B_{e+} + \frac{1}{2} B_{e+} \left[\left(\frac{\partial}{\partial s} - \frac{i}{s}\frac{\partial}{\partial \phi}\right) \left(u_{es} + iu_{e\phi}\right) + \frac{1}{s} \left(u_{es} - i u_{e\phi}\right)\right] \\ 
    &\qquad \qquad \qquad \quad \, + \frac{1}{2} B_{e-} \left[\left(\frac{\partial}{\partial s} + \frac{i}{s}\frac{\partial}{\partial \phi}\right) \left(u_{es} + iu_{e\phi}\right) - \frac{1}{s} \left(u_{es} + i u_{e\phi}\right)\right] \\ 
    \frac{\partial B_{e-}}{\partial t} &= - \left(\mathbf{u}_e \cdot \nabla_e\right) B_{e-} + \frac{1}{2} B_{e+} \left[\left(\frac{\partial}{\partial s} - \frac{i}{s}\frac{\partial}{\partial \phi}\right) \left(u_{es} - iu_{e\phi}\right) - \frac{1}{s} \left(u_{es} - i u_{e\phi}\right)\right] \\ 
    &\qquad \qquad \qquad \quad \, + \frac{1}{2} B_{e-} \left[\left(\frac{\partial}{\partial s} + \frac{i}{s}\frac{\partial}{\partial \phi}\right) \left(u_{es} - iu_{e\phi}\right) + \frac{1}{s} \left(u_{es} + i u_{e\phi}\right)\right] \\ 
    %
    \frac{\partial B_{e+,z}}{\partial t} &= - \left(\mathbf{u}_e \cdot \nabla_e\right) B_{e+,z} + \frac{1}{2} B_{e+,z} \left[\left(\frac{\partial}{\partial s} - \frac{i}{s}\frac{\partial}{\partial \phi}\right) \left(u_{es} + iu_{e\phi}\right) + \frac{1}{s} \left(u_{es} - i u_{e\phi}\right)\right] \\ 
    &\qquad \quad - \frac{\partial u_z}{\partial z} B_{e+,z} + \frac{1}{2} B_{e-,z} \left[\left(\frac{\partial}{\partial s} + \frac{i}{s}\frac{\partial}{\partial \phi}\right) \left(u_{es} + iu_{e\phi}\right) - \frac{1}{s} \left(u_{es} + i u_{e\phi}\right)\right] \\ 
    \frac{\partial B_{e-,z}}{\partial t} &= - \left(\mathbf{u}_e \cdot \nabla_e\right) B_{e-,z} + \frac{1}{2} B_{e+,z} \left[\left(\frac{\partial}{\partial s} - \frac{i}{s}\frac{\partial}{\partial \phi}\right) \left(u_{es} - iu_{e\phi}\right) - \frac{1}{s} \left(u_{es} - i u_{e\phi}\right)\right] \\ 
    &\qquad \quad - \frac{\partial u_z}{\partial z} B_{e-,z} + \frac{1}{2} B_{e-,z} \left[\left(\frac{\partial}{\partial s} + \frac{i}{s}\frac{\partial}{\partial \phi}\right) \left(u_{es} - iu_{e\phi}\right) + \frac{1}{s} \left(u_{es} + i u_{e\phi}\right)\right]
\end{align*}

